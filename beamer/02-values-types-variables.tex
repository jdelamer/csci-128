\documentclass{csbeamer}

\usepackage{hyperref}
\usepackage{listings}
\usepackage{xcolor}

% Python code styling
\lstset{
    language=Python,
    basicstyle=\ttfamily\scriptsize,
    keywordstyle=\color{blue}\bfseries,
    stringstyle=\color{red},
    commentstyle=\color{green!50!black}\itshape,
    showstringspaces=false,
    breaklines=true,
    frame=single,
    numbers=left,
    numberstyle=\tiny\color{gray}
}

% Course information
\title{Values, Types, Variables, and Input}
\course{CSCI 128: Introduction to Computer Science}
\courseshort{CSCI 128}
\term{Winter 2026}
\author{Dr. Jean-Alexis Delamer}

\begin{document}

\begin{frame}
\titlepage
\end{frame}

\begin{frame}{Today's Topics}
    \begin{center}
        \Large Building blocks of Python programs
    \end{center}

    \vspace{1em}

    \textbf{Learning Objectives:}
    \begin{itemize}
        \item<1-> Understand different \textcolor{stfxblue}{data types} in Python
        \item<2-> Learn how to store data in \textcolor{marigold}{variables}
        \item<3-> Use the \textcolor{stfxblue}{assignment operator}
        \item<4-> Get \textcolor{marigold}{input} from users
        \item<5-> Write simple interactive programs
    \end{itemize}

    \vspace{1em}

    \begin{center}
        \onslide<6->{\textit{These are fundamental concepts you'll use in \textcolor{marigold}{\textbf{every program!}}}}
    \end{center}
\end{frame}

\begin{frame}[fragile]{Review: What We Know So Far}
    \begin{lstlisting}
# We can print text
print("Hello, world!")

# We can print numbers and calculations
print(42)
print(5 + 3)

# We can add comments
# This is a comment
    \end{lstlisting}

    \vspace{1em}

    \pause

    \textbf{But we're missing something important...}
    \begin{itemize}
        \item<2-> How do we \textcolor{marigold}{remember} values?
        \item<3-> How do we \textcolor{stfxblue}{reuse} results?
        \item<4-> How do we get \textcolor{marigold}{input} from users?
    \end{itemize}
\end{frame}

\begin{frame}{Values}
    \textcolor{stfxblue}{\textbf{A value}} is a piece of data that a program works with

    \vspace{1em}

    \textbf{Examples of values:}
    \begin{itemize}
        \item<2-> \texttt{42} - a \textcolor{stfxblue}{number}
        \item<3-> \texttt{"Hello"} - \textcolor{marigold}{text}
        \item<4-> \texttt{3.14159} - a \textcolor{stfxblue}{decimal number}
        \item<5-> \texttt{True} - a \textcolor{marigold}{logical value}
    \end{itemize}

    \vspace{1em}

    \onslide<6->{Values are also called \textcolor{stfxblue}{\textbf{literals}} because they literally represent themselves.}

    \vspace{1em}

    \begin{center}
        \onslide<7->{\texttt{42} is literally the number 42}
    \end{center}
\end{frame}

\begin{frame}{Types}
    \textcolor{marigold}{\textbf{Every value has a type}}

    \vspace{1em}

    The \textcolor{stfxblue}{\textbf{type}} determines:
    \begin{itemize}
        \item<2-> What the value represents
        \item<3-> What operations you can perform on it
        \item<4-> How it's stored in memory
    \end{itemize}

    \vspace{1em}

    \begin{center}
        \onslide<5->{\Large \textcolor{stfxblue}{Why do types matter?}}
    \end{center}

    \vspace{1em}

    \onslide<6->{Because computers are very literal! The number \textcolor{stfxblue}{\texttt{5}} and the text \textcolor{marigold}{\texttt{"5"}} are completely different things.}
\end{frame}

\begin{frame}{Common Python Types}
    \begin{center}
        \begin{tabular}{|l|l|l|}
            \hline
            \textbf{Type} & \textbf{Description} & \textbf{Examples} \\
            \hline
            \texttt{int} & Integer (whole number) & \texttt{42, -17, 0, 1000} \\
            \hline
            \texttt{float} & Floating-point (decimal) & \texttt{3.14, -0.5, 2.0} \\
            \hline
            \texttt{str} & String (text) & \texttt{"Hello", 'Python'} \\
            \hline
            \texttt{bool} & Boolean (True/False) & \texttt{True, False} \\
            \hline
        \end{tabular}
    \end{center}

    \vspace{1em}

    \textbf{Note:} Strings must be in quotes (single or double)
\end{frame}

\begin{frame}[fragile]{The type() Function}
    Python can tell you the type of any value:

    \begin{lstlisting}
print(type(42))
print(type(3.14))
print(type("Hello"))
print(type(True))
    \end{lstlisting}

    \vspace{1em}

    \textbf{Output:}
    \begin{verbatim}
<class 'int'>
<class 'float'>
<class 'str'>
<class 'bool'>
    \end{verbatim}

    \vspace{1em}

    Don't worry about the word "class" - it just means "type"
\end{frame}

\begin{frame}[fragile]{Type Matters!}
    \textbf{Same operator, different behavior:}

    \begin{lstlisting}
# Adding numbers
print(5 + 3)        # Output: 8

# Adding strings
print("5" + "3")    # Output: 53

# This causes an error!
print(5 + "3")      # TypeError!
    \end{lstlisting}

    \vspace{1em}

    \begin{center}
        \textcolor{red}{\textbf{Python won't guess what you meant - you must be explicit!}}
    \end{center}
\end{frame}

\begin{frame}[fragile]{Try It: Types}
    \begin{center}
        \Large \textbf{Activity Time!}
    \end{center}

    \vspace{1em}

    \textbf{Use the type() function to check:}
    \begin{enumerate}
        \item What is the type of \texttt{100}?
        \item What is the type of \texttt{100.0}?
        \item What is the type of \texttt{"100"}?
        \item What happens if you try: \texttt{10 * "hello"}?
        \item What about: \texttt{"hello" * 3}?
    \end{enumerate}

    \vspace{1em}

    \textbf{Experiment and see what happens!}
\end{frame}

\begin{frame}{Variables}
    \begin{center}
        \Large \textcolor{marigold}{\textbf{Variables let you store and reuse values}}
    \end{center}

    \vspace{1em}

    Think of a variable as a \textcolor{stfxblue}{labeled box}:
    \begin{itemize}
        \item<2-> The \textcolor{stfxblue}{\textbf{name}} is the label on the box
        \item<3-> The \textcolor{marigold}{\textbf{value}} is what's inside the box
        \item<4-> You can \textcolor{stfxblue}{change} what's in the box anytime
    \end{itemize}

    \vspace{1em}

    \begin{center}
        \onslide<5->{Variables are one of the \textcolor{marigold}{\textbf{most important}} concepts in programming!}
    \end{center}
\end{frame}

\begin{frame}[fragile]{Creating Variables: Assignment}
    Use the \textcolor{stfxblue}{\textbf{assignment operator}} \texttt{=} to create variables:

    \begin{lstlisting}
# Create a variable called 'age' and store 20 in it
age = 20

# Create a variable called 'name' and store "Alice" in it
name = "Alice"

# Create a variable for pi
pi = 3.14159
    \end{lstlisting}

    \vspace{1em}

    \pause

    \textcolor{red}{\textbf{Important:}} The \texttt{=} sign means \textcolor{marigold}{"assign"} not "equals"!
    \begin{itemize}
        \item<3-> Right side is evaluated first
        \item<4-> Result is stored in the variable on the left
    \end{itemize}
\end{frame}

\begin{frame}[fragile]{Using Variables}
    Once created, you can use variables anywhere:

    \begin{lstlisting}
# Create variables
age = 20
name = "Alice"

# Use them in print statements
print(name)
print("Age:", age)

# Use them in calculations
next_year = age + 1
print("Next year:", next_year)
    \end{lstlisting}

    \textbf{Output:}
    \begin{verbatim}
Alice
Age: 20
Next year: 21
    \end{verbatim}
\end{frame}

\begin{frame}[fragile]{Variables Can Change}
    The value in a variable can be updated:

    \begin{lstlisting}
# Start with a value
score = 10
print("Score:", score)

# Change it
score = 20
print("Score:", score)

# Update based on current value
score = score + 5
print("Score:", score)
    \end{lstlisting}

    \textbf{Output:}
    \begin{verbatim}
Score: 10
Score: 20
Score: 25
    \end{verbatim}
\end{frame}

\begin{frame}[fragile]{Understanding Assignment}
    This looks weird but it's valid Python:

    \begin{lstlisting}
x = 5
x = x + 1
print(x)  # Output: 6
    \end{lstlisting}

    \vspace{1em}

    \pause

    \textbf{How it works:}
    \begin{enumerate}
        \item<2-> \texttt{x = 5} - Store 5 in variable x
        \item<3-> \texttt{x = x + 1}:
            \begin{itemize}
                \item Get current value of x (which is 5)
                \item Add 1 to it (5 + 1 = 6)
                \item Store result back in x
            \end{itemize}
        \item<4-> Now x contains 6
    \end{enumerate}

    \vspace{1em}

    \onslide<5->{\textcolor{marigold}{\textbf{Remember:}} Right side is evaluated first, then stored in left side!}
\end{frame}

\begin{frame}{Choosing Variable Names}
    \textcolor{red}{\textbf{Rules (required):}}
    \begin{itemize}
        \item<1-> Must start with a letter or underscore
        \item<2-> Can contain letters, numbers, and underscores
        \item<3-> Cannot be a Python keyword (like \texttt{print}, \texttt{if}, etc.)
        \item<4-> Case-sensitive: \texttt{age} and \texttt{Age} are different
    \end{itemize}

    \vspace{1em}

    \onslide<5->{\textcolor{stfxblue}{\textbf{Conventions (recommended):}}}
    \begin{itemize}
        \item<5-> Use \textcolor{marigold}{descriptive} names: \texttt{student\_count} not \texttt{sc}
        \item<6-> Use lowercase with underscores: \texttt{first\_name}
        \item<7-> Avoid single letters except for simple counters
    \end{itemize}

    \vspace{1em}

    \begin{center}
        \onslide<8->{\textcolor{marigold}{Good names make code easier to understand!}}
    \end{center}
\end{frame}

\begin{frame}[fragile]{Good vs Bad Variable Names}
    \begin{columns}
        \begin{column}{0.5\textwidth}
            \textbf{Good:}
            \begin{lstlisting}
student_name = "Bob"
total_price = 19.99
num_pixels = 1024
is_valid = True
            \end{lstlisting}
        \end{column}

        \begin{column}{0.5\textwidth}
            \textbf{Not so good:}
            \begin{lstlisting}
x = "Bob"
tp = 19.99
n = 1024
flag = True
            \end{lstlisting}
        \end{column}
    \end{columns}

    \vspace{1em}

    \textbf{Ask yourself:} If I read this code in 6 months, will I understand what the variable represents?
\end{frame}

\begin{frame}[fragile]{Try It: Variables}
    \begin{center}
        \Large \textbf{Activity Time!}
    \end{center}

    \vspace{1em}

    \textbf{Write code to:}
    \begin{enumerate}
        \item Store your name in a variable
        \item Store your age in a variable
        \item Calculate your age in 10 years
        \item Print a message: "In 10 years, [name] will be [age]"
    \end{enumerate}

    \vspace{1em}

    \textbf{Bonus:} Can you calculate how many days old you are (roughly)?
\end{frame}

\begin{frame}{Getting Input from Users}
    \begin{center}
        \Large \textcolor{marigold}{Programs are more interesting when they're interactive!}
    \end{center}

    \vspace{1em}

    \textbf{The \textcolor{stfxblue}{\texttt{input()}} function:}
    \begin{itemize}
        \item<2-> Displays a prompt to the user
        \item<3-> Waits for the user to type something
        \item<4-> Returns what the user typed as a \textcolor{marigold}{string}
    \end{itemize}
    \vspace{1em}

    \onslide<5->{This lets users provide data to your program!}
\end{frame}

\begin{frame}[fragile]{Using input()}
    \textbf{Basic syntax:}

    \begin{lstlisting}
variable_name = input("Your prompt here: ")
    \end{lstlisting}

    \vspace{1em}

    \textbf{Example:}
    \begin{lstlisting}
name = input("What is your name? ")
print("Hello,", name)
    \end{lstlisting}

    \vspace{1em}

    \textbf{When you run this:}
    \begin{verbatim}
What is your name? Alice
Hello, Alice
    \end{verbatim}
\end{frame}

\begin{frame}[fragile]{Important: input() Returns a String}
    \textcolor{red}{\textbf{Everything from input() is text!}}

    \begin{lstlisting}
age = input("How old are you? ")
print(type(age))  # <class 'str'>

# This won't work as expected:
next_year = age + 1  # TypeError!
    \end{lstlisting}

    \vspace{1em}

    \pause

    Even if the user types a number, Python treats it as text (a string).

    \vspace{1em}

    \begin{center}
        \onslide<3->{\textcolor{marigold}{\textbf{We need to convert it to a number!}}}
    \end{center}
\end{frame}

\begin{frame}{Type Conversion}
    \textbf{Convert between types using:}
    \begin{itemize}
        \item<1-> \textcolor{stfxblue}{\texttt{int()}} - Convert to integer
        \item<2-> \textcolor{stfxblue}{\texttt{float()}} - Convert to floating-point
        \item<3-> \textcolor{stfxblue}{\texttt{str()}} - Convert to string
    \end{itemize}
\end{frame}

\begin{frame}[fragile]{Type Conversion Example}
    \begin{lstlisting}
# Get input and convert to int
age_text = input("How old are you? ")
age = int(age_text)

# Or do it in one line:
age = int(input("How old are you? "))

# Now we can do math with it
next_year = age + 1
print("Next year you'll be", next_year)
    \end{lstlisting}
\end{frame}

\begin{frame}[fragile]{Type Conversion Examples}
    \begin{lstlisting}
# String to int
int("42")        # Result: 42
int("3.14")      # Error! Can't convert decimal string to int

# String to float
float("3.14")    # Result: 3.14
float("42")      # Result: 42.0

# Number to string
str(42)          # Result: "42"
str(3.14)        # Result: "3.14"

# Int to float and vice versa
float(42)        # Result: 42.0
int(3.14)        # Result: 3 (truncates decimal)
    \end{lstlisting}
\end{frame}

\begin{frame}[fragile]{A Complete Interactive Program}
    \begin{lstlisting}
# Get the user's name
name = input("What is your name? ")

# Get the user's age
age = int(input("How old are you? "))

# Calculate age in 10 years
future_age = age + 10

# Display result
print("Hello,", name)
print("In 10 years, you will be", future_age)
    \end{lstlisting}

    \textbf{This program:}
    \begin{itemize}
        \item Gets input from the user
        \item Stores it in variables
        \item Performs a calculation
        \item Displays the result
    \end{itemize}
\end{frame}

\begin{frame}[fragile]{Try It: Temperature Converter}
    \begin{center}
        \Large \textbf{Activity Time!}
    \end{center}

    \vspace{1em}

    \textbf{Write a program that:}
    \begin{enumerate}
        \item Asks the user for a temperature in Celsius
        \item Converts it to Fahrenheit
        \item Prints the result
    \end{enumerate}

    \vspace{1em}

    \textbf{Formula:} $F = C \times 9/5 + 32$

    \vspace{1em}

    \textbf{Example run:}
    \begin{verbatim}
Enter temperature in Celsius: 0
That is 32.0 degrees Fahrenheit
    \end{verbatim}
\end{frame}

\begin{frame}{Expressions}
    \textbf{An expression is anything that produces a value}

    \vspace{1em}

    \textbf{Examples of expressions:}
    \begin{itemize}
        \item \texttt{5 + 3}
        \item \texttt{x * 2}
        \item \texttt{input("Enter a number: ")}
        \item \texttt{int("42")}
    \end{itemize}

    \vspace{1em}

    \textbf{Expressions can be:}
    \begin{itemize}
        \item Simple literals: \texttt{42}
        \item Variables: \texttt{age}
        \item Operations: \texttt{x + y}
        \item Function calls: \texttt{len("hello")}
        \item Combinations: \texttt{(x + y) * 2}
    \end{itemize}
\end{frame}

\begin{frame}{Statements vs Expressions}
    \textbf{Statement:} An instruction that Python executes
    \begin{itemize}
        \item Assignment: \texttt{x = 5}
        \item Print: \texttt{print("Hello")}
        \item Doesn't necessarily produce a value
    \end{itemize}

    \vspace{1em}

    \textbf{Expression:} Something that evaluates to a value
    \begin{itemize}
        \item \texttt{5 + 3} evaluates to 8
        \item \texttt{x * 2} evaluates to some number
        \item Can be part of a statement
    \end{itemize}

    \vspace{1em}

    \begin{center}
        \texttt{x = 5 + 3} is a statement that contains an expression
    \end{center}
\end{frame}

\begin{frame}[fragile]{Operator Precedence}
    \textbf{Python follows mathematical order of operations:}

    \begin{lstlisting}
print(2 + 3 * 4)      # Output: 14, not 20
print((2 + 3) * 4)    # Output: 20
    \end{lstlisting}

    \vspace{1em}

    \textbf{Order (highest to lowest):}
    \begin{enumerate}
        \item Parentheses: \texttt{()}
        \item Exponentiation: \texttt{**}
        \item Multiplication, Division, Remainder: \texttt{*, /, \%}
        \item Addition, Subtraction: \texttt{+, -}
    \end{enumerate}

    \vspace{1em}

    \textbf{Tip:} Use parentheses to make your intent clear!
\end{frame}

\begin{frame}[fragile]{String Operations}
    Strings have special operations too:

    \begin{lstlisting}
# Concatenation (combining strings)
greeting = "Hello" + " " + "World"
print(greeting)  # Output: Hello World

# Repetition
laugh = "ha" * 3
print(laugh)  # Output: hahahaha

# Length
message = "Python"
print(len(message))  # Output: 6
    \end{lstlisting}

    \vspace{1em}

    \textbf{Note:} \texttt{len()} is a function that returns the length of a string
\end{frame}

\begin{frame}[fragile]{Try It: Mad Libs}
    \textbf{Create a Mad Libs program:}
    \begin{enumerate}
        \item Ask the user for:
            \begin{itemize}
                \item A noun
                \item A verb
                \item An adjective
                \item A number
            \end{itemize}
        \item Use these to create a funny sentence
        \item Print the result
    \end{enumerate}

    \vspace{1em}

    \textbf{Example:} "The [adjective] [noun] decided to [verb] [number] times!"
\end{frame}

\begin{frame}[fragile]{Simple Calculator}
    \textbf{Create a calculator program that:}
    \begin{enumerate}
        \item Asks the user for two numbers
        \item Performs all basic operations (+, -, *, /)
        \item Displays all results
    \end{enumerate}

    \vspace{1em}

    \textbf{Example output:}
    \begin{verbatim}
Enter first number: 10
Enter second number: 3
10 + 3 = 13
10 - 3 = 7
10 * 3 = 30
10 / 3 = 3.3333333333333335
    \end{verbatim}
    \textbf{Bonus:} Also calculate the remainder using \texttt{\%}
\end{frame}

\begin{frame}[fragile]{Circle Calculator}
    \textbf{Write a program that:}
    \begin{enumerate}
        \item Asks the user for the radius of a circle
        \item Calculates the area: $A = \pi r^2$
        \item Calculates the circumference: $C = 2\pi r$
        \item Displays both results
    \end{enumerate}

    \vspace{1em}

    \textbf{Hints:}
    \begin{itemize}
        \item Use \texttt{3.14159} for $\pi$
        \item Use \texttt{**} for exponentiation
    \end{itemize}

    \vspace{1em}

    \textbf{Example:}
    \begin{verbatim}
Enter radius: 5
Area: 78.53975
Circumference: 31.4159
    \end{verbatim}
\end{frame}

\begin{frame}{Key Takeaways}
    \textbf{Today you learned:}
    \begin{itemize}
        \item Python has different types: int, float, str, bool
        \item Variables store values for later use
        \item Use \texttt{=} to assign values to variables
        \item Use \texttt{input()} to get user input
        \item Convert types with \texttt{int()}, \texttt{float()}, \texttt{str()}
        \item Expressions produce values, statements execute actions
    \end{itemize}

    \vspace{1em}

    \begin{center}
        \textbf{These concepts form the foundation of all programming!}
    \end{center}
\end{frame}

\begin{frame}{Common Mistakes to Avoid}
    \begin{enumerate}
        \item Forgetting to convert input to numbers
            \begin{itemize}
                \item \texttt{age = int(input("Age: "))}
            \end{itemize}

        \item Trying to use undefined variables
            \begin{itemize}
                \item Must assign before using!
            \end{itemize}

        \item Mixing types inappropriately
            \begin{itemize}
                \item Can't add string and number directly
            \end{itemize}

        \item Confusing \texttt{=} (assignment) with equality
            \begin{itemize}
                \item \texttt{x = 5} assigns, doesn't test equality
            \end{itemize}
    \end{enumerate}
\end{frame}

\end{document}
