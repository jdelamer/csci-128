\documentclass{csbeamer}

\usepackage{hyperref}
\usepackage{listings}
\usepackage{xcolor}
\usepackage{graphicx}

% Python code styling
\lstset{
    language=Python,
    basicstyle=\ttfamily\small,
    keywordstyle=\color{blue}\bfseries,
    stringstyle=\color{red},
    commentstyle=\color{green!50!black}\itshape,
    showstringspaces=false,
    breaklines=true,
    frame=single,
    numbers=left,
    numberstyle=\tiny\color{gray}
}

% Course information
\title{Text Manipulation and Files}
\course{CSCI 128: Introduction to Computer Science}
\courseshort{CSCI 128}
\term{Winter 2025}
\author{Dr. Jean-Alexis Delamer}

\begin{document}

\begin{frame}
\titlepage
\end{frame}

\begin{frame}{Today's Topics}
    \begin{center}
        \Large Text as another form of media!
    \end{center}

    \vspace{1em}

    \textbf{Learning Objectives:}
    \begin{itemize}
        \item<1-> Process \textcolor{stfxblue}{text} as data
        \item<2-> Read and write \textcolor{marigold}{text files}
        \item<3-> Parse structured text data
        \item<4-> Manipulate strings effectively
        \item<5-> Add text to images (captions, watermarks)
        \item<6-> Create image catalogs with text descriptions
        \item<7-> Process media metadata
    \end{itemize}

    \vspace{1em}

    \begin{center}
        \onslide<8->{\textit{Text is data just like \textcolor{marigold}{images} and \textcolor{stfxblue}{sounds!}}}
    \end{center}
\end{frame}

\begin{frame}{Why Text Processing Matters}
    \textbf{Text in media computation:}

    \vspace{1em}

    \begin{itemize}
        \item<1-> \textcolor{stfxblue}{\textbf{Image metadata:}} Read EXIF data, filenames, captions
        \item<2-> \textcolor{marigold}{\textbf{Batch processing:}} Read lists of files to process
        \item<3-> \textcolor{stfxblue}{\textbf{Watermarks:}} Add text to images
        \item<4-> \textcolor{marigold}{\textbf{Reports:}} Generate descriptions of processed media
        \item<5-> \textcolor{stfxblue}{\textbf{Configuration:}} Read settings from text files
        \item<6-> \textcolor{marigold}{\textbf{Logs:}} Track what processing was done
        \item<7-> \textcolor{stfxblue}{\textbf{Captions:}} Add titles and labels to images
    \end{itemize}

    \vspace{1em}

    \onslide<8->{\textcolor{marigold}{\textbf{Text ties everything together!}}}
\end{frame}

\begin{frame}[fragile]{Review: String Basics}
    \textbf{Strings are sequences of characters:}

    \begin{lstlisting}
# Creating strings
message = "Hello, World!"
filename = 'photo.jpg'
multiline = """This is a
multi-line string"""

# String operations
print(len(message))        # 13
print(message[0])          # H
print(message[7:12])       # World
print(message.upper())     # HELLO, WORLD!
print(message.lower())     # hello, world!

# Checking contents
print("photo" in filename) # True
print(filename.endswith(".jpg"))  # True
    \end{lstlisting}
\end{frame}

\begin{frame}[fragile]{Working with Filenames}
    \textbf{Extract information from filenames:}

    \begin{lstlisting}
import os

filename = "vacation_2024_beach_01.jpg"

# Split into parts
parts = filename.split("_")
theme = parts[0]      # "vacation"
year = parts[1]       # "2024"
location = parts[2]   # "beach"

# Get extension
name, ext = os.path.splitext(filename)
print(name)  # vacation_2024_beach_01
print(ext)   # .jpg

# Build new filename
output = f"processed_{name}{ext}"
print(output)  # processed_vacation_2024_beach_01.jpg
    \end{lstlisting}
\end{frame}

\begin{frame}[fragile]{Reading Text Files}
    \textbf{Read file contents:}

    \begin{lstlisting}
# Read entire file
with open("notes.txt", "r") as file:
    content = file.read()
    print(content)

# Read line by line
with open("notes.txt", "r") as file:
    for line in file:
        line = line.strip()  # Remove newline
        print(line)

# Read all lines into list
with open("notes.txt", "r") as file:
    lines = file.readlines()
    print(f"File has {len(lines)} lines")
    \end{lstlisting}

    \vspace{0.5em}

    \textbf{Always use} \texttt{with} \textbf{to automatically close files!}
\end{frame}

\begin{frame}[fragile]{Writing Text Files}
    \textbf{Create and write to files:}

    \begin{lstlisting}
# Write to file (overwrites existing)
with open("output.txt", "w") as file:
    file.write("First line\n")
    file.write("Second line\n")

# Append to file
with open("log.txt", "a") as file:
    file.write("New log entry\n")

# Write multiple lines
lines = ["Line 1\n", "Line 2\n", "Line 3\n"]
with open("output.txt", "w") as file:
    file.writelines(lines)
    \end{lstlisting}

    \vspace{1em}

    \textbf{Remember:} \texttt{write()} doesn't add newlines automatically!
\end{frame}

\begin{frame}[fragile]{Example: Process List of Images}
    \textbf{Read filenames from text file and process them:}

    \begin{lstlisting}
from PIL import Image

# images.txt contains:
# beach.jpg
# sunset.jpg
# mountain.jpg

with open("images.txt", "r") as file:
    for line in file:
        filename = line.strip()

        try:
            img = Image.open(filename)
            # Apply grayscale
            img = img.convert("L")
            output = f"gray_{filename}"
            img.save(output)
            print(f"Processed: {filename}")
        except FileNotFoundError:
            print(f"Error: {filename} not found")
    \end{lstlisting}
\end{frame}

\begin{frame}[fragile]{Example: Create Processing Log}
    \textbf{Record what was done to each image:}

    \begin{lstlisting}
from PIL import Image
from datetime import datetime

def log_processing(filename, operation, logfile="log.txt"):
    """Add entry to processing log."""
    timestamp = datetime.now().strftime("%Y-%m-%d %H:%M:%S")
    with open(logfile, "a") as file:
        file.write(f"{timestamp} | {filename} | {operation}\n")

# Process image and log it
img = Image.open("photo.jpg")
log_processing("photo.jpg", "Loaded")

img = img.convert("L")
log_processing("photo.jpg", "Converted to grayscale")

img.save("photo_gray.jpg")
log_processing("photo.jpg", "Saved as photo_gray.jpg")
    \end{lstlisting}

    \vspace{0.5em}

    Creates audit trail of all processing!
\end{frame}

\begin{frame}[fragile]{Parsing Structured Text}
    \textbf{Process comma-separated values:}

    \begin{lstlisting}
# data.txt contains:
# filename,filter,brightness
# photo1.jpg,sepia,1.2
# photo2.jpg,grayscale,1.0
# photo3.jpg,vintage,1.5

with open("data.txt", "r") as file:
    lines = file.readlines()
    header = lines[0].strip().split(",")

    for line in lines[1:]:
        parts = line.strip().split(",")
        filename = parts[0]
        filter_type = parts[1]
        brightness = float(parts[2])

        print(f"Apply {filter_type} to {filename}")
        print(f"  Brightness: {brightness}")
    \end{lstlisting}
\end{frame}

\begin{frame}[fragile]{Using CSV Module}
    \textbf{Python's CSV module makes this easier:}

    \begin{lstlisting}
import csv

with open("data.csv", "r") as file:
    reader = csv.DictReader(file)
    for row in reader:
        print(f"Filename: {row['filename']}")
        print(f"Filter: {row['filter']}")
        print(f"Brightness: {row['brightness']}")
        print()

# Writing CSV
data = [
    {"filename": "photo1.jpg", "width": 800, "height": 600},
    {"filename": "photo2.jpg", "width": 1024, "height": 768}
]

with open("images.csv", "w", newline='') as file:
    fieldnames = ["filename", "width", "height"]
    writer = csv.DictWriter(file, fieldnames=fieldnames)
    writer.writeheader()
    writer.writerows(data)
    \end{lstlisting}
\end{frame}

\begin{frame}[fragile]{Example: Image Catalog}
    \textbf{Generate text catalog of images:}

    \begin{lstlisting}
from PIL import Image
import os

def create_catalog(directory, output_file="catalog.txt"):
    """Create text catalog of all images in directory."""
    with open(output_file, "w") as file:
        file.write("IMAGE CATALOG\n")
        file.write("=" * 50 + "\n\n")

        for filename in os.listdir(directory):
            if filename.endswith(('.jpg', '.png', '.gif')):
                path = os.path.join(directory, filename)
                img = Image.open(path)
                width, height = img.size
                mode = img.mode

                file.write(f"File: {filename}\n")
                file.write(f"  Size: {width}x{height}\n")
                file.write(f"  Mode: {mode}\n")
                file.write(f"  Pixels: {width * height:,}\n")
                file.write("\n")

    print(f"Catalog created: {output_file}")

create_catalog("photos")
    \end{lstlisting}
\end{frame}

\begin{frame}[fragile]{Adding Text to Images}
    \textbf{Use Pillow's ImageDraw to add text:}

    \begin{lstlisting}
from PIL import Image, ImageDraw, ImageFont

# Load image
img = Image.open("photo.jpg")
draw = ImageDraw.Draw(img)

# Add text
text = "Summer 2024"
position = (10, 10)
color = (255, 255, 255)  # White

# Simple text (default font)
draw.text(position, text, fill=color)

# Save result
img.save("captioned.jpg")
    \end{lstlisting}

    \vspace{1em}

    \textbf{Common uses:} Watermarks, captions, labels, timestamps
\end{frame}

\begin{frame}[fragile]{Custom Fonts for Text}
    \textbf{Use TrueType fonts for better appearance:}

    \begin{lstlisting}
from PIL import Image, ImageDraw, ImageFont

img = Image.open("photo.jpg")
draw = ImageDraw.Draw(img)

try:
    # Load custom font
    font = ImageFont.truetype("arial.ttf", 36)
except:
    # Fallback to default
    font = ImageFont.load_default()

# Add text with custom font
text = "Copyright 2024"
position = (20, 20)
color = (255, 255, 255)

draw.text(position, text, fill=color, font=font)

img.save("watermarked.jpg")
    \end{lstlisting}
\end{frame}

\begin{frame}[fragile]{Centered Text on Image}
    \textbf{Calculate position to center text:}

    \begin{lstlisting}
from PIL import Image, ImageDraw, ImageFont

img = Image.open("photo.jpg")
draw = ImageDraw.Draw(img)

text = "MY PHOTO"
font = ImageFont.truetype("arial.ttf", 48)

# Get text size
bbox = draw.textbbox((0, 0), text, font=font)
text_width = bbox[2] - bbox[0]
text_height = bbox[3] - bbox[1]

# Calculate centered position
img_width, img_height = img.size
x = (img_width - text_width) // 2
y = (img_height - text_height) // 2

# Draw centered text
draw.text((x, y), text, fill=(255, 255, 255), font=font)

img.save("centered.jpg")
    \end{lstlisting}
\end{frame}

\begin{frame}[fragile]{Text with Outline}
    \textbf{Make text more visible:}

    \begin{lstlisting}
from PIL import Image, ImageDraw, ImageFont

img = Image.open("photo.jpg")
draw = ImageDraw.Draw(img)

text = "ADVENTURE"
position = (50, 50)
font = ImageFont.truetype("arial.ttf", 60)

# Draw outline (black)
outline_color = (0, 0, 0)
for dx in [-2, 0, 2]:
    for dy in [-2, 0, 2]:
        draw.text((position[0]+dx, position[1]+dy),
                  text, fill=outline_color, font=font)

# Draw main text (white)
draw.text(position, text, fill=(255, 255, 255), font=font)

img.save("outlined_text.jpg")
    \end{lstlisting}
\end{frame}

\begin{frame}[fragile]{Example: Timestamp Watermark}
    \textbf{Add timestamp to image:}

    \begin{lstlisting}
from PIL import Image, ImageDraw, ImageFont
from datetime import datetime

def add_timestamp(image, position="bottom-right"):
    """Add current timestamp to image."""
    draw = ImageDraw.Draw(image)

    # Get current time
    timestamp = datetime.now().strftime("%Y-%m-%d %H:%M:%S")

    # Calculate position
    width, height = image.size
    if position == "bottom-right":
        x = width - 200
        y = height - 30
    elif position == "top-left":
        x, y = 10, 10

    # Add text
    draw.text((x, y), timestamp, fill=(255, 255, 255))

    return image

img = Image.open("photo.jpg")
img = add_timestamp(img)
img.save("timestamped.jpg")
    \end{lstlisting}
\end{frame}

\begin{frame}[fragile]{Text Processing: Word Count}
    \textbf{Analyze text files:}

    \begin{lstlisting}
def analyze_text_file(filename):
    """Count words, lines, and characters."""
    with open(filename, "r") as file:
        content = file.read()
        lines = content.split("\n")
        words = content.split()

        print(f"File: {filename}")
        print(f"Characters: {len(content)}")
        print(f"Lines: {len(lines)}")
        print(f"Words: {len(words)}")

        # Most common words
        word_counts = {}
        for word in words:
            word = word.lower().strip(".,!?")
            word_counts[word] = word_counts.get(word, 0) + 1

        # Top 5 words
        sorted_words = sorted(word_counts.items(),
                             key=lambda x: x[1], reverse=True)
        print("\nTop 5 words:")
        for word, count in sorted_words[:5]:
            print(f"  {word}: {count}")

analyze_text_file("notes.txt")
    \end{lstlisting}
\end{frame}

\begin{frame}[fragile]{Configuration Files}
    \textbf{Read settings from text file:}

    \begin{lstlisting}
# config.txt contains:
# brightness=1.5
# contrast=1.2
# output_format=png
# watermark_text=Copyright 2024

def load_config(filename):
    """Load configuration from text file."""
    config = {}
    with open(filename, "r") as file:
        for line in file:
            line = line.strip()
            if line and not line.startswith("#"):
                key, value = line.split("=")
                config[key.strip()] = value.strip()
    return config

config = load_config("config.txt")
brightness = float(config.get("brightness", 1.0))
contrast = float(config.get("contrast", 1.0))
output_format = config.get("output_format", "jpg")
watermark = config.get("watermark_text", "")

print(f"Brightness: {brightness}")
print(f"Contrast: {contrast}")
print(f"Format: {output_format}")
print(f"Watermark: {watermark}")
    \end{lstlisting}
\end{frame}

\begin{frame}[fragile]{Example: Batch Processor with Instructions}
    \textbf{Read processing instructions from file:}

    \begin{lstlisting}
# instructions.txt:
# beach.jpg grayscale brighten
# sunset.jpg sepia darken
# mountain.jpg negative

from PIL import Image
import image_filters

with open("instructions.txt", "r") as file:
    for line in file:
        parts = line.strip().split()
        filename = parts[0]
        operations = parts[1:]

        print(f"Processing {filename}:")
        img = Image.open(filename)

        for op in operations:
            if op == "grayscale":
                img = image_filters.grayscale(img)
            elif op == "brighten":
                img = image_filters.brighten(img, 30)
            elif op == "darken":
                img = image_filters.darken(img, 30)
            print(f"  Applied: {op}")

        output = f"processed_{filename}"
        img.save(output)
        print(f"  Saved: {output}\n")
    \end{lstlisting}
\end{frame}

\begin{frame}[fragile]{Creating Reports}
    \textbf{Generate HTML report of processed images:}

    \begin{lstlisting}
def create_html_report(images, output="report.html"):
    """Create HTML report with image previews."""
    with open(output, "w") as file:
        file.write("<html><head><title>Image Report</title></head>\n")
        file.write("<body>\n")
        file.write("<h1>Image Processing Report</h1>\n")

        for img_info in images:
            file.write(f"<div style='margin: 20px;'>\n")
            file.write(f"<h2>{img_info['filename']}</h2>\n")
            file.write(f"<img src='{img_info['filename']}' ")
            file.write(f"width='400'><br>\n")
            file.write(f"Size: {img_info['width']}x{img_info['height']}<br>\n")
            file.write(f"Filter: {img_info['filter']}<br>\n")
            file.write("</div>\n")

        file.write("</body></html>")

images = [
    {"filename": "photo1.jpg", "width": 800,
     "height": 600, "filter": "grayscale"},
    {"filename": "photo2.jpg", "width": 1024,
     "height": 768, "filter": "sepia"}
]

create_html_report(images)
    \end{lstlisting}
\end{frame}

\begin{frame}[fragile]{Error Handling with Files}
    \textbf{Handle file errors gracefully:}

    \begin{lstlisting}
def safe_read_file(filename):
    """Read file with error handling."""
    try:
        with open(filename, "r") as file:
            return file.read()
    except FileNotFoundError:
        print(f"Error: {filename} not found")
        return None
    except PermissionError:
        print(f"Error: No permission to read {filename}")
        return None
    except Exception as e:
        print(f"Error reading {filename}: {e}")
        return None

content = safe_read_file("data.txt")
if content is not None:
    print("File contents:")
    print(content)
else:
    print("Could not read file")
    \end{lstlisting}
\end{frame}

\begin{frame}{File Paths and Organization}
    \textbf{Organize your project files:}

    \vspace{1em}

    \textbf{Good structure:}
    \begin{itemize}
        \item \texttt{input/} - Original images
        \item \texttt{output/} - Processed images
        \item \texttt{data/} - Text files, CSV files
        \item \texttt{logs/} - Processing logs
        \item \texttt{config/} - Configuration files
        \item \texttt{reports/} - Generated reports
    \end{itemize}

    \vspace{1em}

    \textbf{Use} \texttt{os.path.join()} \textbf{for paths:}
    \begin{itemize}
        \item Works on all operating systems
        \item Handles path separators correctly
        \item Example: \texttt{os.path.join("output", "processed.jpg")}
    \end{itemize}
\end{frame}

\begin{frame}[fragile]{Example: Complete Media Processor}
    \textbf{Putting it all together:}

    \begin{lstlisting}
from PIL import Image
import image_filters
import os

def process_with_log(input_dir, output_dir, log_file):
    """Process images and create log."""
    # Create output directory if needed
    if not os.path.exists(output_dir):
        os.makedirs(output_dir)

    with open(log_file, "w") as log:
        log.write("Processing Log\n")
        log.write("=" * 50 + "\n\n")

        for filename in os.listdir(input_dir):
            if filename.endswith(('.jpg', '.png')):
                input_path = os.path.join(input_dir, filename)
                output_path = os.path.join(output_dir,
                                          f"processed_{filename}")

                img = Image.open(input_path)
                img = image_filters.grayscale(img)
                img.save(output_path)

                log.write(f"Processed: {filename}\n")
                log.write(f"  Input: {input_path}\n")
                log.write(f"  Output: {output_path}\n\n")

process_with_log("input", "output", "processing_log.txt")
    \end{lstlisting}
\end{frame}

\begin{frame}{Practice Problems}
    \textbf{Try these:}

    \begin{enumerate}
        \item Read a list of filenames from a text file and rename them all with a prefix

        \item Create a function that adds captions to multiple images from a CSV file

        \item Write a program that creates an inventory of all images in a directory

        \item Build a watermark tool that reads text and position from a config file

        \item Create a batch processor that reads instructions from a text file

        \item Generate an HTML gallery page with thumbnails and descriptions
    \end{enumerate}
\end{frame}

\begin{frame}{Summary}
    \textbf{Key concepts:}
    \begin{itemize}
        \item Text is data just like images and sounds
        \item Read files with \texttt{open(filename, "r")}
        \item Write files with \texttt{open(filename, "w")}
        \item Always use \texttt{with} statement
        \item Parse structured data with \texttt{split()} or CSV module
        \item Add text to images with ImageDraw
        \item Create logs and reports for tracking
        \item Handle file errors gracefully
        \item Organize files into logical directories
    \end{itemize}

    \vspace{1em}

    \textbf{Next time:} Web Data and Information
\end{frame}

\end{document}
