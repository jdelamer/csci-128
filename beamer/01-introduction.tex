\documentclass{csbeamer}

\usepackage{hyperref}
\usepackage{listings}
\usepackage{xcolor}

% Python code styling
\lstset{
    language=Python,
    basicstyle=\ttfamily\scriptsize,
    keywordstyle=\color{blue}\bfseries,
    stringstyle=\color{red},
    commentstyle=\color{green!50!black}\itshape,
    showstringspaces=false,
    breaklines=true,
    frame=single,
    numbers=left,
    numberstyle=\tiny\color{gray}
}

% Course information
\title{Introduction to Computing and Programming}
\course{CSCI 128: Introduction to Computer Science}
\courseshort{CSCI 128}
\term{Winter 2025}
\author{Dr. Jean-Alexis Delamer}

\begin{document}

\begin{frame}
\titlepage
\end{frame}

\begin{frame}{Welcome to CSCI 128!}
    \begin{center}
        \Large What will you learn in this course?
    \end{center}

    \vspace{1em}

    \begin{itemize}
        \item<1-> How to write computer programs in \textcolor{stfxblue}{\textbf{Python}}
        \item<2-> How to manipulate media: \textcolor{marigold}{pictures, sounds, and text}
        \item<3-> How computers represent and process information
        \item<4-> How to solve problems \textcolor{stfxblue}{computationally}
        \item<5-> A foundation in computer science thinking
    \end{itemize}

    \vspace{1em}

    \begin{center}
        \onslide<6->{\textcolor{marigold}{\textbf{This is a skill that will distinguish you in any field!}}}
    \end{center}
\end{frame}

\begin{frame}{About This Course}
    \textbf{Why \textcolor{marigold}{Media Computation}?}

    \begin{itemize}
        \item Traditional programming courses start with abstract concepts
        \item We'll start by doing \textcolor{stfxblue}{interesting things} with pictures and sounds
        \item More engaging and motivating
        \item Same programming concepts, but with \textcolor{marigold}{visible results}
    \end{itemize}

    \vspace{1em}

    \pause

    \textbf{What You Need:}
    \begin{itemize}
        \item Curiosity and willingness to experiment
        \item Persistence (programming requires trial and error)
        \item No prior programming experience needed!
    \end{itemize}
\end{frame}

\begin{frame}{What is a Computer?}
    \begin{columns}
        \begin{column}{0.5\textwidth}
            \textbf{A computer is:}
            \begin{itemize}
                \item A machine that processes information
                \item Follows precise instructions
                \item Stores and retrieves data
                \item Performs calculations
            \end{itemize}
        \end{column}

        \begin{column}{0.5\textwidth}
            \textbf{Types of computers:}
            \begin{itemize}
                \item Desktop/Laptop
                \item Smartphone
                \item Embedded systems
                \item Servers
                \item Supercomputers
            \end{itemize}
        \end{column}
    \end{columns}

    \vspace{1em}

    \begin{center}
        \textit{What computers all have in common: They execute programs!}
    \end{center}
\end{frame}

\begin{frame}{What is a Program?}
    \textbf{A program is:}
    \begin{itemize}
        \item A sequence of instructions that tells a computer what to do
        \item Written in a programming language
        \item Can be executed (run) by the computer
    \end{itemize}

    \vspace{1em}

    \textbf{Programming languages are formal languages:}
    \begin{itemize}
        \item Precise syntax (grammar rules)
        \item No ambiguity allowed
        \item Unlike natural languages (English, French, etc.)
    \end{itemize}

    \vspace{1em}

    \begin{center}
        \textcolor{marigold}{\Large\textbf{Think of it like a recipe, but the chef is very literal!}}
    \end{center}
\end{frame}

\begin{frame}{Why Python?}
    \begin{columns}
        \begin{column}{0.5\textwidth}
            \textbf{Advantages:}
            \begin{itemize}
                \item<1-> \textcolor{stfxblue}{Easy to read and write}
                \item<2-> Powerful and versatile
                \item<3-> Huge community and libraries
                \item<4-> Used in industry and research
                \item<5-> \textcolor{marigold}{Great for beginners}
            \end{itemize}
        \end{column}

        \begin{column}{0.5\textwidth}
            \textbf{Used for:}
            \begin{itemize}
                \item<6-> Web development
                \item<7-> \textcolor{stfxblue}{Data science}
                \item<8-> \textcolor{stfxblue}{Machine learning}
                \item<9-> Scientific computing
                \item<10-> Automation
                \item<11-> \textcolor{marigold}{\textbf{Media processing!}}
            \end{itemize}
        \end{column}
    \end{columns}

    \vspace{1em}

    \begin{center}
        \onslide<12->{Python is one of the most popular programming languages in the world!}
    \end{center}
\end{frame}

\begin{frame}{How Computers Store Information}
    \textbf{\textcolor{stfxblue}{Binary System:}}
    \begin{itemize}
        \item Computers store everything as \textcolor{marigold}{\textbf{0s and 1s}} (bits)
        \item Everything: numbers, text, pictures, sounds, videos
        \item \textcolor{stfxblue}{8 bits = 1 byte}
    \end{itemize}

    \vspace{1em}

    \pause

    \textbf{Key Components:}
    \begin{itemize}
        \item<2-> \textcolor{stfxblue}{\textbf{CPU (Processor):}} Does the calculations and executes instructions
        \item<3-> \textcolor{marigold}{\textbf{RAM (Memory):}} Temporary storage while programs run
        \item<4-> \textcolor{stfxblue}{\textbf{Storage (Hard Drive/SSD):}} Permanent storage for files
        \item<5-> \textbf{Input/Output:} Keyboard, mouse, screen, etc.
    \end{itemize}
\end{frame}

\begin{frame}[fragile]{Your First Python Program}
    \begin{center}
        \Large Let's write the traditional first program!
    \end{center}

    \vspace{2em}

    \pause

    \begin{lstlisting}
print("Hello, world!")
    \end{lstlisting}

    \vspace{2em}

    \pause

    \textbf{What this does:}
    \begin{itemize}
        \item<3-> \textcolor{stfxblue}{\texttt{print()}} is a function that displays output
        \item<4-> The text in quotes is called a \textcolor{marigold}{\textit{string}}
        \item<5-> When you run this, it displays: \texttt{Hello, world!}
    \end{itemize}
\end{frame}

\begin{frame}[fragile]{The print() Function}
    \textbf{Basic usage:}

    \begin{lstlisting}
print("Welcome to CSCI 128")
print("This is fun!")
print("I am learning Python")
    \end{lstlisting}

    \vspace{1em}

    \textbf{Output:}
    \begin{verbatim}
Welcome to CSCI 128
This is fun!
I am learning Python
    \end{verbatim}

    \vspace{1em}

    \begin{itemize}
        \item Each \texttt{print()} statement outputs on a new line
        \item Text must be in quotes (single or double)
    \end{itemize}
\end{frame}

\begin{frame}[fragile]{Try It: Print Statements}
    \begin{center}
        \Large \textbf{Activity Time!}
    \end{center}

    \vspace{1em}

    \textbf{Task:} Write Python code to print:
    \begin{enumerate}
        \item Your name
        \item Your favorite hobby
        \item Why you're taking this course
    \end{enumerate}

    \vspace{1em}

    \textbf{Example:}
    \begin{lstlisting}
print("My name is Alice")
print("I love painting")
print("I want to learn programming")
    \end{lstlisting}
\end{frame}

\begin{frame}{What is Computation?}
    \textcolor{stfxblue}{\textbf{Computation}} is the process of calculating or processing information

    \vspace{1em}

    \textbf{Computers can:}
    \begin{itemize}
        \item<2-> Perform mathematical operations
        \item<3-> Compare values
        \item<4-> Store and retrieve data
        \item<5-> Make decisions based on conditions
        \item<6-> Repeat operations many times
    \end{itemize}

    \vspace{1em}

    \begin{center}
        \onslide<7->{\textit{Programming is about telling the computer how to perform computations to solve problems}}
    \end{center}
\end{frame}

\begin{frame}{Using Python as a Calculator}
    Python can perform arithmetic operations:

    \vspace{1em}

    \begin{columns}
        \begin{column}{0.5\textwidth}
            \textbf{Operations:}
            \begin{itemize}
                \item \textcolor{stfxblue}{\texttt{+}} Addition
                \item \textcolor{stfxblue}{\texttt{-}} Subtraction
                \item \textcolor{stfxblue}{\texttt{*}} Multiplication
                \item \textcolor{stfxblue}{\texttt{/}} Division
                \item \textcolor{marigold}{\texttt{**}} Exponentiation
                \item \textcolor{marigold}{\texttt{\%}} Remainder
            \end{itemize}
        \end{column}

        \begin{column}{0.5\textwidth}
            \textbf{Examples:}
            \begin{itemize}
                \item \texttt{5 + 3} → 8
                \item \texttt{10 - 4} → 6
                \item \texttt{6 * 7} → 42
                \item \texttt{15 / 3} → 5.0
                \item \texttt{2 ** 3} → 8
                \item \texttt{17 \% 5} → 2
            \end{itemize}
        \end{column}
    \end{columns}
\end{frame}

\begin{frame}[fragile]{Printing Calculations}
    You can print the results of calculations:

    \begin{lstlisting}
print(5 + 3)
print(10 * 4)
print(100 / 4)
print(2 ** 10)
    \end{lstlisting}

    \vspace{1em}

    \textbf{Output:}
    \begin{verbatim}
8
40
25.0
1024
    \end{verbatim}

    \vspace{1em}

    Notice: No quotes needed for numbers!
\end{frame}

\begin{frame}[fragile]{Combining Text and Numbers}
    You can print multiple things at once:

    \begin{lstlisting}
print("The answer is:", 42)
print("5 + 3 =", 5 + 3)
print("2 to the power of 8 is", 2**8)
    \end{lstlisting}

    \vspace{1em}

    \textbf{Output:}
    \begin{verbatim}
The answer is: 42
5 + 3 = 8
2 to the power of 8 is 256
    \end{verbatim}

    \vspace{1em}

    Separate items with commas in the \texttt{print()} function
\end{frame}

\begin{frame}[fragile]{Try It: Calculator}
    \begin{center}
        \Large \textbf{Activity Time!}
    \end{center}

    \vspace{1em}

    \textbf{Calculate and print:}
    \begin{enumerate}
        \item How many hours are in a week?
        \item How many seconds are in a day?
        \item What is $15^3$?
        \item What is the remainder when 100 is divided by 7?
    \end{enumerate}

    \vspace{1em}

    \textbf{Hint:} Think about what operations you need!
\end{frame}

\begin{frame}[fragile]{Comments in Code}
    \textbf{Comments} are notes in your code that Python ignores

    \vspace{1em}

    \begin{lstlisting}
# This is a comment
print("Hello")  # Comments can go after code too

# Use comments to explain what your code does
# This helps you and others understand it later
    \end{lstlisting}

    \vspace{1em}

    \textbf{Best practices:}
    \begin{itemize}
        \item Use \texttt{\#} to start a comment
        \item Explain \textit{why} you're doing something, not just \textit{what}
        \item Keep comments clear and concise
    \end{itemize}
\end{frame}

\begin{frame}{Errors are Normal!}
    \begin{center}
        \Large \textcolor{red}{\textbf{You WILL make mistakes.}} \textcolor{stfxblue}{\textbf{That's okay!}}
    \end{center}

    \vspace{1em}

    \pause

    \textbf{Common beginner errors:}
    \begin{itemize}
        \item Forgetting quotation marks around text
        \item Misspelling \texttt{print}
        \item Forgetting parentheses
        \item Mismatched quotes
    \end{itemize}

    \vspace{1em}

    \pause

    \textbf{When you get an error:}
    \begin{itemize}
        \item \textcolor{stfxblue}{Read the error message carefully}
        \item Check the line number indicated
        \item Look for typos
        \item \textcolor{marigold}{Don't panic! Errors are learning opportunities}
    \end{itemize}
\end{frame}

\begin{frame}[fragile]{Common Error Example}
    \textbf{Error:}
    \begin{lstlisting}
print(Hello, world!)
    \end{lstlisting}

    \textbf{Error message:}
    \begin{verbatim}
NameError: name 'Hello' is not defined
    \end{verbatim}

    \vspace{1em}

    \textbf{Problem:} Missing quotes around the text!

    \vspace{1em}

    \textbf{Fix:}
    \begin{lstlisting}
print("Hello, world!")
    \end{lstlisting}
\end{frame}

\begin{frame}{The Programming Process}
    \begin{enumerate}
        \item<1-> \textbf{Understand the problem}
            \begin{itemize}
                \item What are you trying to accomplish?
            \end{itemize}

        \item<2-> \textbf{Plan your solution}
            \begin{itemize}
                \item What steps are needed?
            \end{itemize}

        \item<3-> \textbf{Write the code}
            \begin{itemize}
                \item Translate your plan into Python
            \end{itemize}

        \item<4-> \textbf{Test and debug}
            \begin{itemize}
                \item Run it, find errors, fix them
            \end{itemize}

        \item<5-> \textbf{Refine and improve}
            \begin{itemize}
                \item Make it better, clearer, more efficient
            \end{itemize}
    \end{enumerate}
\end{frame}

\begin{frame}{Key Takeaways}
    \textbf{Today you learned:}
    \begin{itemize}
        \item<1-> What computers and programs are
        \item<2-> Why we're learning \textcolor{stfxblue}{Python}
        \item<3-> How to use \textcolor{stfxblue}{\texttt{print()}} to display output
        \item<4-> How to perform calculations in Python
        \item<5-> How to write comments
        \item<6-> That \textcolor{marigold}{errors are normal} and part of learning!
    \end{itemize}

    \vspace{1em}

    \begin{center}
        \onslide<7->{\textcolor{marigold}{\Large\textbf{You are now a programmer!}}}
    \end{center}
\end{frame}

\begin{frame}{For Next Class}
    \textbf{Before next session:}
    \begin{itemize}
        \item Make sure your Python environment is set up
        \item Experiment with \texttt{print()} statements
        \item Try different calculations
        \item Read Chapter 1 of the textbook
    \end{itemize}

    \vspace{1em}

    \textbf{Next session:}
    \begin{itemize}
        \item Values and types
        \item Variables
        \item User input
        \item Your first lab!
    \end{itemize}

    \vspace{1em}

    \begin{center}
        \textbf{Questions?}
    \end{center}
\end{frame}

\end{document}
