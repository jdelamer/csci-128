\documentclass{csbeamer}

\usepackage{hyperref}
\usepackage{listings}
\usepackage{xcolor}
\usepackage{graphicx}

% Python code styling
\lstset{
    language=Python,
    basicstyle=\ttfamily\small,
    keywordstyle=\color{blue}\bfseries,
    stringstyle=\color{red},
    commentstyle=\color{green!50!black}\itshape,
    showstringspaces=false,
    breaklines=true,
    frame=single,
    numbers=left,
    numberstyle=\tiny\color{gray}
}

% Course information
\title{Web Data and Information}
\course{CSCI 128: Introduction to Computer Science}
\courseshort{CSCI 128}
\term{Winter 2025}
\author{Dr. Jean-Alexis Delamer}

\begin{document}

\begin{frame}
\titlepage
\end{frame}

\begin{frame}{Today's Topics}
    \begin{center}
        \Large Working with structured data!
    \end{center}

    \vspace{1em}

    \textbf{Learning Objectives:}
    \begin{itemize}
        \item<1-> Work with \textcolor{stfxblue}{CSV data files}
        \item<2-> Process structured image metadata
        \item<3-> Create \textcolor{marigold}{data-driven} media applications
        \item<4-> Generate reports combining text and images
        \item<5-> Organize and analyze collections of media
        \item<6-> Build image galleries from data
    \end{itemize}

    \vspace{1em}

    \begin{center}
        \onslide<7->{\textit{\textcolor{marigold}{Data} + \textcolor{stfxblue}{Media} = \textcolor{marigold}{\textbf{Powerful applications!}}}}
    \end{center}
\end{frame}

\begin{frame}[fragile]{CSV Files Review}
    \textcolor{stfxblue}{\textbf{CSV = Comma-Separated Values}}

    \vspace{1em}

    \begin{lstlisting}[numbers=none]
filename,width,height,filter,date
beach.jpg,1920,1080,sepia,2024-01-15
sunset.jpg,1024,768,grayscale,2024-01-16
mountain.jpg,800,600,vintage,2024-01-17
    \end{lstlisting}

    \vspace{1em}

    \textbf{Perfect for:}
    \begin{itemize}
        \item<2-> \textcolor{marigold}{Image metadata}
        \item<3-> Processing instructions
        \item<4-> Batch operations
        \item<5-> Reports and catalogs
    \end{itemize}
\end{frame}

\begin{frame}[fragile]{Reading CSV with Python}
    \begin{lstlisting}
import csv

with open("images.csv", "r") as file:
    reader = csv.DictReader(file)
    for row in reader:
        filename = row['filename']
        width = int(row['width'])
        height = int(row['height'])
        filter_type = row['filter']

        print(f"{filename}: {width}x{height}, {filter_type}")
    \end{lstlisting}

    \vspace{1em}

    \textcolor{stfxblue}{\texttt{DictReader}} gives us dictionaries with column names as keys!
\end{frame}

\begin{frame}[fragile]{Example: Batch Process from CSV}
    \begin{lstlisting}
import csv
from PIL import Image
import image_filters

with open("batch.csv", "r") as file:
    reader = csv.DictReader(file)
    for row in reader:
        filename = row['filename']
        filter_type = row['filter']
        brightness = float(row['brightness'])

        # Process image
        img = Image.open(filename)

        if filter_type == "grayscale":
            img = image_filters.grayscale(img)
        elif filter_type == "sepia":
            img = image_filters.sepia(img)

        img = image_filters.brighten(img, int(brightness * 50))

        output = f"processed_{filename}"
        img.save(output)
        print(f"Saved {output}")
    \end{lstlisting}
\end{frame}

\begin{frame}[fragile]{Creating CSV Reports}
    \begin{lstlisting}
import csv
from PIL import Image
import os

def create_image_report(directory, output_csv):
    """Create CSV report of all images."""
    with open(output_csv, "w", newline='') as file:
        fieldnames = ['filename', 'width', 'height',
                     'mode', 'size_kb']
        writer = csv.DictWriter(file, fieldnames=fieldnames)
        writer.writeheader()

        for filename in os.listdir(directory):
            if filename.endswith(('.jpg', '.png', '.gif')):
                path = os.path.join(directory, filename)
                img = Image.open(path)
                size_kb = os.path.getsize(path) / 1024

                writer.writerow({
                    'filename': filename,
                    'width': img.size[0],
                    'height': img.size[1],
                    'mode': img.mode,
                    'size_kb': f"{size_kb:.2f}"
                })

create_image_report("photos", "report.csv")
    \end{lstlisting}
\end{frame}

\begin{frame}[fragile]{JSON Data Format}
    \textcolor{marigold}{\textbf{JSON = JavaScript Object Notation}}

    \begin{lstlisting}
{
  "title": "My Photo Gallery",
  "images": [
    {
      "filename": "beach.jpg",
      "caption": "Summer vacation",
      "tags": ["beach", "sunset", "ocean"]
    },
    {
      "filename": "mountain.jpg",
      "caption": "Hiking trip",
      "tags": ["mountain", "nature"]
    }
  ]
}
    \end{lstlisting}

    \vspace{0.5em}

    \onslide<2->{JSON is great for \textcolor{stfxblue}{hierarchical} data!}
\end{frame}

\begin{frame}[fragile]{Reading JSON in Python}
    \begin{lstlisting}
import json

with open("gallery.json", "r") as file:
    data = json.load(file)

title = data['title']
print(f"Gallery: {title}")

for img_info in data['images']:
    filename = img_info['filename']
    caption = img_info['caption']
    tags = ", ".join(img_info['tags'])

    print(f"\n{filename}")
    print(f"  Caption: {caption}")
    print(f"  Tags: {tags}")
    \end{lstlisting}
\end{frame}

\begin{frame}[fragile]{Creating HTML Galleries}
    \begin{lstlisting}
def create_html_gallery(images, output="gallery.html"):
    """Generate HTML gallery from image list."""
    with open(output, "w") as file:
        file.write("<html><head>\n")
        file.write("<title>Photo Gallery</title>\n")
        file.write("<style>")
        file.write("img { width: 300px; margin: 10px; }")
        file.write("</style>\n")
        file.write("</head><body>\n")
        file.write("<h1>My Photo Gallery</h1>\n")

        for img_info in images:
            file.write("<div>\n")
            file.write(f"<img src='{img_info['file']}'><br>\n")
            file.write(f"<b>{img_info['title']}</b><br>\n")
            file.write(f"{img_info['description']}<br>\n")
            file.write("</div>\n")

        file.write("</body></html>")

images = [
    {"file": "beach.jpg", "title": "Beach",
     "description": "Summer 2024"},
    {"file": "sunset.jpg", "title": "Sunset",
     "description": "Beautiful evening"}
]

create_html_gallery(images)
    \end{lstlisting}
\end{frame}

\begin{frame}{Practice Problems}
    \textbf{Try these:}

    \begin{enumerate}
        \item Create a CSV file listing all images in a directory with their dimensions

        \item Read processing instructions from CSV and batch process images

        \item Generate an HTML photo album from a JSON configuration file

        \item Create a program that finds all images larger than a certain size

        \item Build a tagging system using CSV to store image tags

        \item Generate statistics about your image collection
    \end{enumerate}
\end{frame}

\begin{frame}{Summary}
    \textbf{Key concepts:}
    \begin{itemize}
        \item CSV files store tabular data
        \item JSON stores hierarchical data
        \item Use csv module for reading/writing CSV
        \item Use json module for JSON files
        \item Combine data with media for powerful applications
        \item Generate reports and galleries from data
        \item Data-driven processing enables automation
    \end{itemize}

    \vspace{1em}

    \textbf{Next time:} Animation and Course Review
\end{frame}

\end{document}
