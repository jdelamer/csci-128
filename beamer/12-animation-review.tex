\documentclass{csbeamer}

\usepackage{hyperref}
\usepackage{listings}
\usepackage{xcolor}
\usepackage{graphicx}

% Python code styling
\lstset{
    language=Python,
    basicstyle=\ttfamily\small,
    keywordstyle=\color{blue}\bfseries,
    stringstyle=\color{red},
    commentstyle=\color{green!50!black}\itshape,
    showstringspaces=false,
    breaklines=true,
    frame=single,
    numbers=left,
    numberstyle=\tiny\color{gray}
}

% Course information
\title{Animation and Course Review}
\course{CSCI 128: Introduction to Computer Science}
\courseshort{CSCI 128}
\term{Winter 2025}
\author{Dr. Jean-Alexis Delamer}

\begin{document}

\begin{frame}
\titlepage
\end{frame}

\begin{frame}{Today's Topics}
    \begin{center}
        \Large From still images to animation!
    \end{center}

    \vspace{1em}

    \textbf{Learning Objectives:}
    \begin{itemize}
        \item<1-> Create simple \textcolor{marigold}{animations} from image sequences
        \item<2-> Understand \textcolor{stfxblue}{frame-based} animation
        \item<3-> Review key course concepts
        \item<4-> Connect everything we've learned
        \item<5-> Plan your final project
        \item<6-> Look ahead to future learning
    \end{itemize}

    \vspace{1em}

    \begin{center}
        \onslide<7->{\textit{\textcolor{marigold}{\textbf{Bringing it all together!}}}}
    \end{center}
\end{frame}

\begin{frame}{Animation Basics}
    \textcolor{marigold}{\textbf{Animation = Sequence of images displayed rapidly}}

    \vspace{1em}

    \textbf{Key concepts:}
    \begin{itemize}
        \item<1-> \textcolor{stfxblue}{\textbf{Frame:}} One image in the sequence
        \item<2-> \textcolor{marigold}{\textbf{Frame rate:}} Images per second (FPS)
        \item<3-> \textbf{24 FPS:} Film standard
        \item<4-> \textbf{30 FPS:} Video standard
        \item<5-> \textbf{60 FPS:} Smooth video games
    \end{itemize}

    \vspace{1em}

    \onslide<6->{\textcolor{marigold}{\textbf{Illusion of motion from still images!}}}
\end{frame}

\begin{frame}[fragile]{Creating Animation Frames}
    \begin{lstlisting}
from PIL import Image

def create_fade_animation(img, num_frames=10):
    """Create frames that fade image to black."""
    frames = []

    for i in range(num_frames):
        factor = 1.0 - (i / num_frames)
        frame = img.copy()
        pixels = frame.load()
        width, height = frame.size

        for x in range(width):
            for y in range(height):
                r, g, b = pixels[x, y]
                pixels[x, y] = (
                    int(r * factor),
                    int(g * factor),
                    int(b * factor)
                )

        frames.append(frame)
        frame.save(f"frame_{i:03d}.jpg")

    return frames

img = Image.open("photo.jpg")
create_fade_animation(img)
    \end{lstlisting}
\end{frame}

\begin{frame}[fragile]{Creating GIF Animations}
    \begin{lstlisting}
from PIL import Image

def create_gif(frames, output="animation.gif", duration=100):
    """Save frames as animated GIF.

    duration: milliseconds per frame
    """
    frames[0].save(
        output,
        save_all=True,
        append_images=frames[1:],
        duration=duration,
        loop=0  # 0 = loop forever
    )

# Create animation
img = Image.open("photo.jpg")
frames = create_fade_animation(img, num_frames=20)
create_gif(frames, "fade.gif", duration=50)
    \end{lstlisting}

    \vspace{0.5em}

    Creates a looping \textcolor{marigold}{animated GIF!}
\end{frame}

\begin{frame}{Course Review: What We've Learned}
    \textbf{Week 1-2: \textcolor{marigold}{Images}}
    \begin{itemize}
        \item<1-> Python basics, variables, functions
        \item<2-> Digital images as pixel data
        \item<3-> RGB color model
        \item<4-> Basic filters and effects
    \end{itemize}

    \vspace{0.5em}

    \onslide<5->{\textbf{Week 3: Advanced Images}}
    \begin{itemize}
        \item<5-> Transformations (rotate, mirror, scale)
        \item<6-> Collages and compositing
        \item<7-> Position-based effects
    \end{itemize}

    \vspace{0.5em}

    \onslide<8->{\textbf{Week 4: \textcolor{stfxblue}{Sound}}}
    \begin{itemize}
        \item<8-> Digital sound as samples
        \item<9-> Sound effects (echo, mixing)
        \item<10-> Generating tones
    \end{itemize}
\end{frame}

\begin{frame}{Course Review: What We've Learned (cont.)}
    \textbf{Week 5: \textcolor{marigold}{Program Organization}}
    \begin{itemize}
        \item<1-> Functions and modules
        \item<2-> Building libraries
        \item<3-> Testing and debugging
    \end{itemize}

    \vspace{0.5em}

    \onslide<4->{\textbf{Week 6: Text and Files}}
    \begin{itemize}
        \item<4-> File I/O operations
        \item<5-> Adding text to images
        \item<6-> Creating logs and reports
    \end{itemize}

    \vspace{0.5em}

    \onslide<7->{\textbf{Week 7: \textcolor{stfxblue}{Data and Animation}}}
    \begin{itemize}
        \item<7-> CSV and JSON data
        \item<8-> Data-driven applications
        \item<9-> Creating animations
    \end{itemize}
\end{frame}

\begin{frame}{Key Programming Concepts}
    \textbf{You now understand:}

    \vspace{1em}

    \begin{itemize}
        \item<1-> \textcolor{stfxblue}{\textbf{Variables:}} Store and manipulate data
        \item \textbf{Functions:} Organize and reuse code
        \item \textbf{Loops:} Repeat operations efficiently
        \item \textbf{Conditionals:} Make decisions in code
        \item \textbf{Data structures:} Lists, dictionaries, strings
        \item \textbf{Files:} Persist data between runs
        \item \textbf{Modules:} Organize larger programs
        \item \textbf{Debugging:} Find and fix problems
    \end{itemize}

    \vspace{1em}

    These concepts apply to ANY programming language!
\end{frame}

\begin{frame}{Media Computation Skills}
    \textbf{You can now:}

    \vspace{1em}

    \begin{itemize}
        \item Load, modify, and save images
        \item Apply filters and transformations
        \item Create collages and compositions
        \item Process digital sound
        \item Add text and watermarks to images
        \item Batch process multiple files
        \item Generate reports and catalogs
        \item Create simple animations
        \item Build complete media applications
    \end{itemize}

    \vspace{1em}

    \textbf{Real skills for real projects!}
\end{frame}

\begin{frame}{Final Project Ideas}
    \textbf{Put it all together:}

    \vspace{1em}

    \begin{itemize}
        \item Photo filter application with multiple effects
        \item Automated image watermarking tool
        \item Image gallery generator
        \item Sound effect library and mixer
        \item Batch image processor with CSV configuration
        \item Animated meme generator
        \item Photo collage maker
        \item Image metadata analyzer
        \item Custom Instagram-style filter pack
        \item Slideshow creator with transitions
    \end{itemize}

    \vspace{1em}

    Choose something YOU'RE interested in!
\end{frame}

\begin{frame}{Project Requirements}
    \textbf{Your final project should:}

    \vspace{1em}

    \begin{enumerate}
        \item Use functions to organize code
        \item Process images or sound (or both!)
        \item Read from or write to files
        \item Include error handling
        \item Be well-documented
        \item Demonstrate multiple concepts from class
        \item Actually work!
    \end{enumerate}

    \vspace{1em}

    \textbf{Aim for useful and creative!}
\end{frame}

\begin{frame}{Beyond This Course}
    \textbf{Continue learning:}

    \vspace{1em}

    \textbf{More Python:}
    \begin{itemize}
        \item Object-oriented programming
        \item Web development (Flask, Django)
        \item Data science (pandas, NumPy)
        \item Machine learning (scikit-learn)
    \end{itemize}

    \vspace{0.5em}

    \textbf{More media:}
    \begin{itemize}
        \item OpenCV for computer vision
        \item MoviePy for video editing
        \item Advanced audio with librosa
        \item 3D graphics with PyGame
    \end{itemize}

    \vspace{0.5em}

    \textbf{Other languages:}
    \begin{itemize}
        \item JavaScript for web
        \item Java or C++ for systems
        \item SQL for databases
    \end{itemize}
\end{frame}

\begin{frame}{Resources}
    \textbf{Keep learning:}

    \vspace{1em}

    \textbf{Documentation:}
    \begin{itemize}
        \item Python.org official documentation
        \item Pillow documentation
        \item Pydub documentation
    \end{itemize}

    \vspace{0.5em}

    \textbf{Online learning:}
    \begin{itemize}
        \item Real Python tutorials
        \item Python tutorials on YouTube
        \item Coursera/edX Python courses
    \end{itemize}

    \vspace{0.5em}

    \textbf{Practice:}
    \begin{itemize}
        \item Build personal projects
        \item Contribute to open source
        \item Join coding communities
    \end{itemize}
\end{frame}

\begin{frame}{Final Advice}
    \textbf{To succeed in programming:}

    \vspace{1em}

    \begin{enumerate}
        \item \textbf{Practice regularly:} Code every day
        \item \textbf{Build projects:} Learning by doing
        \item \textbf{Read others' code:} Learn from examples
        \item \textbf{Ask questions:} No question is stupid
        \item \textbf{Debug systematically:} Don't guess
        \item \textbf{Start simple:} Then add complexity
        \item \textbf{Have fun:} Enjoy creating things!
    \end{enumerate}

    \vspace{1em}

    \textbf{You are now a programmer!}
\end{frame}

\begin{frame}{Thank You!}
    \begin{center}
        \LARGE Thank you for a great course!

        \vspace{2em}

        \large Questions about final projects?

        \vspace{2em}

        \normalsize
        Good luck with your projects!

        \vspace{1em}

        Keep coding and creating!
    \end{center}
\end{frame}

\end{document}
