\documentclass{csbeamer}

\usepackage{hyperref}
\usepackage{listings}
\usepackage{xcolor}
\usepackage{graphicx}

% Python code styling
\lstset{
    language=Python,
    basicstyle=\ttfamily\small,
    keywordstyle=\color{blue}\bfseries,
    stringstyle=\color{red},
    commentstyle=\color{green!50!black}\itshape,
    showstringspaces=false,
    breaklines=true,
    frame=single,
    numbers=left,
    numberstyle=\tiny\color{gray}
}

% Course information
\title{Combining Images and Collages}
\course{CSCI 128: Introduction to Computer Science}
\courseshort{CSCI 128}
\term{Winter 2025}
\author{Dr. Jean-Alexis Delamer}

\begin{document}

\begin{frame}
\titlepage
\end{frame}

\begin{frame}{Today's Topics}
    \begin{center}
        \Large Creating art from multiple images!
    \end{center}

    \vspace{1em}

    \textbf{Learning Objectives:}
    \begin{itemize}
        \item<1-> Copy regions from one image to another
        \item<2-> Paste images onto \textcolor{stfxblue}{canvases}
        \item<3-> Create \textcolor{marigold}{photo collages}
        \item<4-> Blend images together
        \item<5-> Apply \textcolor{stfxblue}{chromakey} (green screen) effects
        \item<6-> Build complex compositions
    \end{itemize}

    \vspace{1em}

    \begin{center}
        \onslide<7->{\textit{Combine everything you've learned to create \textcolor{marigold}{\textbf{amazing compositions!}}}}
    \end{center}
\end{frame}

\begin{frame}{Review: Our Toolkit So Far}
    \textbf{We can now:}
    \begin{itemize}
        \item<1-> Load and save images
        \item<2-> Modify pixel \textcolor{marigold}{colors} (filters)
        \item<3-> Transform images (mirror, rotate, crop, scale)
        \item<4-> Process all pixels systematically
    \end{itemize}

    \vspace{1em}

    \pause

    \textbf{Today we add:}
    \begin{itemize}
        \item<6-> Working with \textcolor{stfxblue}{multiple images}
        \item<7-> Copying regions between images
        \item<8-> Creating \textcolor{marigold}{compositions}
        \item<9-> Advanced effects
    \end{itemize}
\end{frame}

\begin{frame}{Pasting One Image onto Another}
    \textbf{The basic operation:}

    \vspace{1em}

    \begin{enumerate}
        \item<1-> Start with a \textcolor{stfxblue}{canvas} (background image)
        \item<2-> Load a source image
        \item<3-> Choose where to place it
        \item<4-> Copy pixels from source to canvas
    \end{enumerate}

    \vspace{1em}

    \pause

    \textbf{Key decisions:}
    \begin{itemize}
        \item<6-> What size should the canvas be?
        \item<7-> Where should we place the image?
        \item<8-> What if the image goes off the edge?
        \item<9-> Should we scale the image first?
    \end{itemize}
\end{frame}

\begin{frame}[fragile]{Creating a Canvas}
    \begin{lstlisting}
from PIL import Image

# Create a blank white canvas
canvas = Image.new("RGB", (800, 600), (255, 255, 255))

# Or a black canvas
canvas = Image.new("RGB", (800, 600), (0, 0, 0))

# Or a colored canvas
canvas = Image.new("RGB", (800, 600), (200, 220, 255))

# Or load an existing image as the base
canvas = Image.open("background.jpg")

canvas.show()
    \end{lstlisting}

    \vspace{1em}

    \textbf{The \textcolor{stfxblue}{canvas} is the foundation of your composition}
\end{frame}

\begin{frame}[fragile]{Pasting an Image: Manual Method}
    \begin{lstlisting}
from PIL import Image

def paste_image(canvas, source, x_offset, y_offset):
    canvas_pixels = canvas.load()
    source_pixels = source.load()
    src_width, src_height = source.size

    for x in range(src_width):
        for y in range(src_height):
            # Calculate destination position
            dest_x = x_offset + x
            dest_y = y_offset + y

            # Check bounds
            if (0 <= dest_x < canvas.size[0] and
                0 <= dest_y < canvas.size[1]):
                canvas_pixels[dest_x, dest_y] = \
                    source_pixels[x, y]

    return canvas
    \end{lstlisting}
\end{frame}

\begin{frame}[fragile]{Using Pillow's Paste Method}
    \textbf{Pillow provides a convenient paste() method:}

    \begin{lstlisting}
from PIL import Image

# Load images
canvas = Image.new("RGB", (800, 600), (255, 255, 255))
photo = Image.open("photo.jpg")

# Paste at position (100, 50)
canvas.paste(photo, (100, 50))

canvas.save("composition.jpg")
canvas.show()
    \end{lstlisting}

    \vspace{1em}

    \pause

    \textbf{Parameters:}
    \begin{itemize}
        \item<2-> First: Image to paste
        \item<3-> Second: \textcolor{marigold}{(x, y) position} for top-left corner
        \item<4-> Optional third: Mask for transparency
    \end{itemize}
\end{frame}

\begin{frame}[fragile]{Try It: Simple Collage}
    \begin{center}
        \Large \textbf{Activity Time!}
    \end{center}

    \vspace{1em}

    \textbf{Create a simple 2-photo collage:}
    \begin{enumerate}
        \item Create a white 800×400 canvas
        \item Load two photos
        \item Scale them to 350×400 (or similar)
        \item Paste one on the left side
        \item Paste one on the right side
        \item Save the result
    \end{enumerate}

    \vspace{1em}

    \textbf{Hint:} Calculate positions to center them with a gap
\end{frame}

\begin{frame}{Creating a Grid Collage}
    \textbf{Grid layouts are popular:}

    \vspace{1em}

    \begin{itemize}
        \item 2×2 grid: 4 images
        \item 3×3 grid: 9 images
        \item 2×3 grid: 6 images
    \end{itemize}

    \vspace{1em}

    \textbf{Algorithm:}
    \begin{enumerate}
        \item Decide grid dimensions (rows × columns)
        \item Calculate cell size
        \item Calculate canvas size
        \item Scale images to fit cells
        \item Place each image in its grid position
    \end{enumerate}
\end{frame}

\begin{frame}[fragile]{2×2 Grid Collage}
    \begin{lstlisting}
from PIL import Image

# Load and scale images
img1 = Image.open("photo1.jpg").resize((300, 300))
img2 = Image.open("photo2.jpg").resize((300, 300))
img3 = Image.open("photo3.jpg").resize((300, 300))
img4 = Image.open("photo4.jpg").resize((300, 300))

# Create canvas
canvas = Image.new("RGB", (600, 600))

# Paste in grid positions
canvas.paste(img1, (0, 0))      # Top-left
canvas.paste(img2, (300, 0))    # Top-right
canvas.paste(img3, (0, 300))    # Bottom-left
canvas.paste(img4, (300, 300))  # Bottom-right

canvas.save("grid_collage.jpg")
canvas.show()
    \end{lstlisting}
\end{frame}

\begin{frame}[fragile]{Generalized Grid Collage}
    \begin{lstlisting}
from PIL import Image

def create_grid_collage(images, rows, cols, cell_size):
    canvas_width = cols * cell_size
    canvas_height = rows * cell_size
    canvas = Image.new("RGB", (canvas_width, canvas_height))

    for i, img in enumerate(images):
        if i >= rows * cols:
            break

        # Calculate grid position
        row = i // cols
        col = i % cols
        x = col * cell_size
        y = row * cell_size

        # Scale and paste
        scaled = img.resize((cell_size, cell_size))
        canvas.paste(scaled, (x, y))

    return canvas
    \end{lstlisting}
\end{frame}

\begin{frame}{Adding Borders and Spacing}
    \textbf{Make collages more polished:}

    \vspace{1em}

    \begin{itemize}
        \item Add padding between images
        \item Add borders around images
        \item Use a colored background
    \end{itemize}

    \vspace{1em}

    \textbf{Approach:}
    \begin{itemize}
        \item Make canvas larger to account for gaps
        \item Calculate positions with spacing
        \item Optionally draw borders
    \end{itemize}
\end{frame}

\begin{frame}[fragile]{Grid Collage with Spacing}
    \begin{lstlisting}
def create_grid_with_spacing(images, rows, cols,
                             cell_size, spacing):
    canvas_width = cols * cell_size + (cols + 1) * spacing
    canvas_height = rows * cell_size + (rows + 1) * spacing
    canvas = Image.new("RGB", (canvas_width, canvas_height),
                       (240, 240, 240))  # Light gray

    for i, img in enumerate(images):
        if i >= rows * cols:
            break

        row = i // cols
        col = i % cols
        x = (col + 1) * spacing + col * cell_size
        y = (row + 1) * spacing + row * cell_size

        scaled = img.resize((cell_size, cell_size))
        canvas.paste(scaled, (x, y))

    return canvas
    \end{lstlisting}
\end{frame}

\begin{frame}{Blending Images}
    \textbf{Blending combines two images:}

    \vspace{1em}

    \begin{itemize}
        \item Both images contribute to the result
        \item Creates a "double exposure" effect
        \item Controlled by an alpha (blend factor)
    \end{itemize}

    \vspace{1em}

    \textbf{Formula:}
    \begin{center}
        $\text{result} = \alpha \times \text{img1} + (1-\alpha) \times \text{img2}$
    \end{center}

    \vspace{1em}

    \begin{itemize}
        \item $\alpha = 0.5$: Equal blend (50/50)
        \item $\alpha = 0.7$: More of img1
        \item $\alpha = 0.3$: More of img2
    \end{itemize}
\end{frame}

\begin{frame}[fragile]{Implementing Image Blending}
    \begin{lstlisting}
from PIL import Image

def blend_images(img1, img2, alpha):
    # Ensure same size
    width, height = img1.size
    img2 = img2.resize((width, height))

    result = Image.new("RGB", (width, height))
    pix1 = img1.load()
    pix2 = img2.load()
    result_pix = result.load()

    for x in range(width):
        for y in range(height):
            r1, g1, b1 = pix1[x, y]
            r2, g2, b2 = pix2[x, y]

            r = int(alpha * r1 + (1 - alpha) * r2)
            g = int(alpha * g1 + (1 - alpha) * g2)
            b = int(alpha * b1 + (1 - alpha) * b2)

            result_pix[x, y] = (r, g, b)

    return result
    \end{lstlisting}
\end{frame}

\begin{frame}[fragile]{Using Pillow's Blend Method}
    \textbf{Pillow has a built-in blend function:}

    \begin{lstlisting}
from PIL import Image

img1 = Image.open("photo1.jpg")
img2 = Image.open("photo2.jpg")

# Ensure same size
img2 = img2.resize(img1.size)

# Blend with 50/50 mix
blended = Image.blend(img1, img2, 0.5)

# More of img2 (alpha = 0.7 means 70% img2, 30% img1)
blended2 = Image.blend(img1, img2, 0.7)

blended.save("blended.jpg")
blended.show()
    \end{lstlisting}

    \vspace{1em}

    \textbf{Note:} Higher alpha = more of the second image
\end{frame}

\begin{frame}[fragile]{Try It: Blending Effects}
    \begin{center}
        \Large \textbf{Activity Time!}
    \end{center}

    \vspace{1em}

    \textbf{Experiment with blending:}
    \begin{enumerate}
        \item Choose two different images
        \item Blend them at alpha = 0.5
        \item Try alpha = 0.25 and alpha = 0.75
        \item Which looks best?
        \item Try blending a photo with a solid color image
    \end{enumerate}

    \vspace{1em}

    \textbf{Challenge:} Can you create a gradient blend where alpha varies across the image?
\end{frame}

\begin{frame}{Chromakey (Green Screen)}
    \textbf{Chromakey replaces a specific color:}

    \vspace{1em}

    \begin{itemize}
        \item Used in movies and TV
        \item Film subject in front of green/blue screen
        \item Replace that color with a different background
        \item Creates illusion of different location
    \end{itemize}

    \vspace{1em}

    \textbf{Algorithm:}
    \begin{enumerate}
        \item For each pixel in foreground image
        \item If pixel is close to the key color (green)
        \item Copy pixel from background instead
        \item Otherwise, copy pixel from foreground
    \end{enumerate}
\end{frame}

\begin{frame}[fragile]{Simple Chromakey Implementation}
    \begin{lstlisting}
from PIL import Image

def chromakey(foreground, background, key_color,
              threshold=50):
    fg_pixels = foreground.load()
    bg_pixels = background.load()
    result = Image.new("RGB", foreground.size)
    result_pixels = result.load()

    for x in range(foreground.size[0]):
        for y in range(foreground.size[1]):
            r, g, b = fg_pixels[x, y]

            # Calculate distance to key color
            distance = abs(r - key_color[0]) + \
                      abs(g - key_color[1]) + \
                      abs(b - key_color[2])

            if distance < threshold:
                result_pixels[x, y] = bg_pixels[x, y]
            else:
                result_pixels[x, y] = (r, g, b)

    return result
    \end{lstlisting}
\end{frame}

\begin{frame}[fragile]{Using Chromakey}
    \begin{lstlisting}
from PIL import Image

# Load images
foreground = Image.open("person_greenscreen.jpg")
background = Image.open("beach.jpg")

# Ensure same size
background = background.resize(foreground.size)

# Apply chromakey
# Green screen color (approximately)
green = (0, 255, 0)
result = chromakey(foreground, background, green,
                   threshold=100)

result.save("composited.jpg")
result.show()
    \end{lstlisting}

    \vspace{1em}

    \textbf{Note:} Threshold controls how close to green is replaced
\end{frame}

\begin{frame}{Improving Chromakey}
    \textbf{Challenges with simple chromakey:}

    \vspace{1em}

    \begin{itemize}
        \item Hard edges look unnatural
        \item Shadows on green screen
        \item Green spill on subject
        \item Not all greens are the same
    \end{itemize}

    \vspace{1em}

    \textbf{Improvements:}
    \begin{itemize}
        \item Use HSV color space instead of RGB
        \item Add edge feathering (gradual transparency)
        \item Adjust threshold per-region
        \item Color correction for spill
    \end{itemize}

    \vspace{1em}

    \textit{Professional compositing is complex, but the basics are here!}
\end{frame}

\begin{frame}{Overlaying with Transparency}
    \textbf{Some images have transparent regions:}

    \vspace{1em}

    \begin{itemize}
        \item PNG format supports alpha channel
        \item Alpha = 0: Fully transparent
        \item Alpha = 255: Fully opaque
        \item In between: Partially transparent
    \end{itemize}

    \vspace{1em}

    \textbf{Mode "RGBA":}
    \begin{itemize}
        \item R = Red (0-255)
        \item G = Green (0-255)
        \item B = Blue (0-255)
        \item A = Alpha (0-255)
    \end{itemize}
\end{frame}

\begin{frame}[fragile]{Pasting with Transparency}
    \begin{lstlisting}
from PIL import Image

# Load base image
canvas = Image.open("background.jpg")

# Load image with transparency (PNG)
overlay = Image.open("logo.png")

# Paste using the alpha channel as a mask
# The third parameter tells it to use transparency
canvas.paste(overlay, (100, 50), overlay)

canvas.save("with_overlay.jpg")
canvas.show()
    \end{lstlisting}

    \vspace{1em}

    \textbf{The third parameter (overlay) uses its alpha channel as a mask}

    \vspace{0.5em}

    Transparent pixels won't overwrite the canvas!
\end{frame}

\begin{frame}{Lab: Create a Photo Collage}
    \begin{center}
        \Large \textbf{Main Lab Activity!}
    \end{center}

    \vspace{1em}

    \textbf{Build a complete collage program that:}
    \begin{enumerate}
        \item Takes a list of image filenames
        \item Creates a grid layout (you choose dimensions)
        \item Scales all images to the same size
        \item Adds spacing between images
        \item Adds a title (use PIL's text capabilities or separate)
        \item Saves the final result
    \end{enumerate}

    \vspace{1em}

    \textbf{Requirements:}
    \begin{itemize}
        \item Use at least 4 images
        \item Include error handling for missing files
        \item Make it reusable (parameters for grid size, etc.)
    \end{itemize}
\end{frame}

\begin{frame}[fragile]{Bonus: Adding Text to Images}
    \begin{lstlisting}
from PIL import Image, ImageDraw, ImageFont

# Create or load image
img = Image.new("RGB", (400, 200), (255, 255, 255))

# Create drawing context
draw = ImageDraw.Draw(img)

# Add text
text = "My Photo Collage"
position = (50, 80)
color = (0, 0, 0)  # Black

# Use default font
draw.text(position, text, fill=color)

# Or specify a font
# font = ImageFont.truetype("arial.ttf", 36)
# draw.text(position, text, fill=color, font=font)

img.show()
    \end{lstlisting}
\end{frame}

\begin{frame}{Lab Extensions}
    \textbf{Additional challenges:}

    \vspace{1em}

    \begin{enumerate}
        \item \textbf{Polaroid style:}
            \begin{itemize}
                \item Add white borders around each photo
                \item Slight rotation for each image
            \end{itemize}

        \item \textbf{Gradient background:}
            \begin{itemize}
                \item Create a gradient-colored canvas
            \end{itemize}

        \item \textbf{Blended montage:}
            \begin{itemize}
                \item Blend overlapping images with varying alpha
            \end{itemize}

        \item \textbf{Picture-in-picture:}
            \begin{itemize}
                \item One large background, smaller images overlaid
            \end{itemize}
    \end{enumerate}
\end{frame}

\begin{frame}{Creative Ideas}
    \textbf{What else can you create?}

    \vspace{1em}

    \begin{itemize}
        \item \textbf{Before/After:} Show image transformations
        \item \textbf{Comic strip:} Sequential images with borders
        \item \textbf{Photo booth:} Multiple shots in strip layout
        \item \textbf{Meme templates:} Images with text overlays
        \item \textbf{Contact sheet:} Many thumbnail images
        \item \textbf{Desktop wallpaper:} Blend photos artistically
        \item \textbf{Social media:} Profile banners, posts
    \end{itemize}

    \vspace{1em}

    \begin{center}
        \textit{The only limit is your creativity!}
    \end{center}
\end{frame}

\begin{frame}{Best Practices for Compositions}
    \textbf{Tips for great results:}

    \vspace{1em}

    \begin{enumerate}
        \item \textbf{Plan your layout first}
            \begin{itemize}
                \item Sketch on paper
                \item Calculate dimensions
            \end{itemize}

        \item \textbf{Consider aspect ratios}
            \begin{itemize}
                \item Crop images to same ratio
                \item Or embrace different shapes
            \end{itemize}

        \item \textbf{Use consistent spacing}
            \begin{itemize}
                \item Makes it look professional
            \end{itemize}

        \item \textbf{Pay attention to resolution}
            \begin{itemize}
                \item Don't scale up too much (pixelation)
                \item Start with high-quality sources
            \end{itemize}

        \item \textbf{Save in appropriate format}
            \begin{itemize}
                \item JPEG for photos, PNG for graphics
            \end{itemize}
    \end{enumerate}
\end{frame}

\begin{frame}[fragile]{Performance Considerations}
    \textbf{Working with multiple images:}

    \vspace{1em}

    \begin{itemize}
        \item Multiple large images use lots of memory
        \item Scale images down before combining when possible
        \item Process images one at a time
        \item Close/delete images you're done with
        \item Test with small images first
    \end{itemize}

    \vspace{1em}

    \begin{lstlisting}
# Good practice
img = Image.open("huge_photo.jpg")
img = img.resize((800, 600))  # Scale first
# Now work with smaller image
    \end{lstlisting}
\end{frame}

\begin{frame}{Key Takeaways}
    \textbf{Today you learned:}
    \begin{itemize}
        \item How to paste images onto canvases
        \item How to create grid collages
        \item How to add spacing and borders
        \item How to blend images together
        \item How to implement chromakey (green screen)
        \item How to work with transparency
        \item How to combine all techniques for creative compositions
    \end{itemize}

    \vspace{1em}

    \begin{center}
        \textbf{You can now create complex image compositions!}
    \end{center}
\end{frame}

\begin{frame}{Real-World Applications}
    \textbf{What you've learned is used in:}

    \vspace{1em}

    \begin{itemize}
        \item \textbf{Social media apps:} Instagram, Snapchat layouts
        \item \textbf{Photo editing:} Photoshop, GIMP
        \item \textbf{Film production:} Special effects, compositing
        \item \textbf{E-commerce:} Product photography, catalogs
        \item \textbf{Marketing:} Advertisements, posters
        \item \textbf{Web design:} Image galleries, portfolios
        \item \textbf{Game development:} Sprite sheets, textures
    \end{itemize}

    \vspace{1em}

    \begin{center}
        These are professional, industry-standard techniques!
    \end{center}
\end{frame}

\begin{frame}{For Next Class}
    \textbf{Before next session:}
    \begin{itemize}
        \item Complete your photo collage lab
        \item Experiment with different layouts
        \item Try the bonus challenges
        \item Read Chapter 5 of the textbook
        \item Think about: How could we automate repetitive tasks?
    \end{itemize}

    \vspace{1em}

    \textbf{Next session:}
    \begin{itemize}
        \item Introduction to functions (formal)
        \item Defining your own functions
        \item Parameters and return values
        \item Building reusable code libraries
        \item Organizing your image processing code
    \end{itemize}

    \vspace{1em}

    \begin{center}
        \textbf{Questions?}
    \end{center}
\end{frame}

\end{document}
