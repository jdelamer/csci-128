\documentclass{csexam}

\usepackage{listings}
\usepackage{xcolor}

% Python code styling
\lstset{
    language=Python,
    basicstyle=\ttfamily\small,
    keywordstyle=\color{blue}\bfseries,
    stringstyle=\color{red},
    commentstyle=\color{green!50!black}\itshape,
    showstringspaces=false,
    breaklines=true,
    frame=single,
    numbers=left,
    numberstyle=\tiny\color{gray}
}

% Course information
\course{CSCI 128: Introduction to Computer Science}
\courseshort{CSCI 128}
\exam{Quiz 4: Picture Filters}
\examdate{Winter 2025}

\begin{document}

\maketitle

\begin{rules}
\textbf{Instructions:}
\begin{itemize}
    \item This quiz should take approximately 15 minutes.
    \item Answer all questions in the space provided.
    \item Show your work where applicable.
    \item No calculators or electronic devices allowed.
\end{itemize}
\end{rules}

\showgrade

\begin{questions}

\question[5] Write a function that converts an image to grayscale using the luminance formula: \texttt{gray = 0.299*R + 0.587*G + 0.114*B}

\vspace{2.5in}

\question[3] What does a negative filter do to an image? Write the formula to calculate the negative of each color channel.

\vspace{1.2in}

\question[4] Complete the code to increase the brightness of every pixel by 50:

\begin{lstlisting}
def brighten(img, amount):
    width, height = img.size
    pixels = img.load()

    for x in range(width):
        for y in range(height):
            r, g, b = pixels[x, y]
            # Complete this part
            new_r = _________________
            new_g = _________________
            new_b = _________________
            pixels[x, y] = (new_r, new_g, new_b)

    return img
\end{lstlisting}

\question[4] A pixel has RGB values (200, 100, 50). What will be the RGB values after applying:

\begin{parts}
\part A negative filter?

\vspace{0.5in}

\part A brightness increase of 30?

\vspace{0.5in}

\part Setting all values to their average (simple grayscale)?

\vspace{0.5in}
\end{parts}

\question[4] Explain why we need to use \texttt{max(0, min(255, value))} or similar clamping when modifying pixel values.

\vspace{1.5in}

\end{questions}

\end{document}
