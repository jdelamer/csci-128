\documentclass{csexam}

\usepackage{listings}
\usepackage{xcolor}

% Python code styling
\lstset{
    language=Python,
    basicstyle=\ttfamily\small,
    keywordstyle=\color{blue}\bfseries,
    stringstyle=\color{red},
    commentstyle=\color{green!50!black}\itshape,
    showstringspaces=false,
    breaklines=true,
    frame=single,
    numbers=left,
    numberstyle=\tiny\color{gray}
}

% Course information
\course{CSCI 128: Introduction to Computer Science}
\courseshort{CSCI 128}
\exam{Quiz 8}
\examdate{Winter 2025}

\begin{document}

\maketitle

\begin{rules}
\textbf{Instructions:}
\begin{itemize}
    \item This quiz covers material from Lecture 11 (Web Data and Structured Data)
    \item Answer all questions in the space provided
    \item Show your work where applicable
    \item Time limit: 15 minutes
\end{itemize}
\end{rules}

\vspace{0.2in}

\showgrade

\newpage

\begin{questions}

\question[2] What does CSV stand for? Why is CSV a popular format for storing structured data?
\vspace{1.5in}

\question[3] Given the following CSV data:
\begin{lstlisting}[numbers=none]
name,age,city
Alice,25,Boston
Bob,30,Seattle
\end{lstlisting}
If you read this file and split the first data line by commas, what would be at index 2?
\vspace{1in}

\question[3] Write Python code to read a CSV file called \texttt{images.csv} and print the first column of each row (skip the header).
\vspace{1.5in}

\question[2] What is the \texttt{csv} module in Python used for? Name one advantage of using it instead of manually splitting strings.
\vspace{1.5in}

\question[3] If you have a list of dictionaries containing image metadata (filename, width, height), write code to create a text report listing each filename and its dimensions.

\vspace{1.5in}

\question[2] What is the purpose of combining structured data (like CSV) with media processing? Give one practical example.
\vspace{1.5in}

\end{questions}

\end{document}
