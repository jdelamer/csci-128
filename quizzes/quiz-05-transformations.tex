\documentclass{csexam}

\usepackage{listings}
\usepackage{xcolor}

% Python code styling
\lstset{
    language=Python,
    basicstyle=\ttfamily\small,
    keywordstyle=\color{blue}\bfseries,
    stringstyle=\color{red},
    commentstyle=\color{green!50!black}\itshape,
    showstringspaces=false,
    breaklines=true,
    frame=single,
    numbers=left,
    numberstyle=\tiny\color{gray}
}

% Course information
\course{CSCI 128: Introduction to Computer Science}
\courseshort{CSCI 128}
\exam{Quiz 5: Picture Transformations}
\examdate{Winter 2025}

\begin{document}

\maketitle

\begin{rules}
\textbf{Instructions:}
\begin{itemize}
    \item This quiz should take approximately 15 minutes.
    \item Answer all questions in the space provided.
    \item Show your work where applicable.
    \item No calculators or electronic devices allowed.
\end{itemize}
\end{rules}

\showgrade

\begin{questions}

\question[5] For an image with width=100 and height=80, what are the new coordinates after each transformation?

\begin{parts}
\part Horizontal mirror: pixel at (10, 20) maps to (\underline{\hspace{0.5in}}, \underline{\hspace{0.5in}})

Formula: \underline{\hspace{2in}}

\vspace{0.3in}

\part Vertical mirror: pixel at (30, 15) maps to (\underline{\hspace{0.5in}}, \underline{\hspace{0.5in}})

Formula: \underline{\hspace{2in}}

\vspace{0.3in}

\part 90° clockwise rotation: pixel at (10, 20) maps to (\underline{\hspace{0.5in}}, \underline{\hspace{0.5in}})

What are the dimensions of the rotated image? \underline{\hspace{1in}} × \underline{\hspace{1in}}

\vspace{0.3in}
\end{parts}

\question[4] Complete the code to mirror an image horizontally:

\begin{lstlisting}
def mirror_horizontal(img):
    width, height = img.size
    mirrored = Image.new("RGB", (width, height))

    src_pixels = img.load()
    dst_pixels = mirrored.load()

    for x in range(width):
        for y in range(height):
            color = src_pixels[x, y]
            dst_pixels[______________, y] = color

    return mirrored
\end{lstlisting}

\question[3] Write code to crop a 200×150 region starting at position (50, 30) from an image:

\vspace{2in}

\question[4] True or False:

\begin{parts}
\part Applying a horizontal mirror twice returns the original image. \underline{\hspace{0.8in}}

\part Rotating an image 90° clockwise changes its dimensions. \underline{\hspace{0.8in}}

\part Cropping always produces an image smaller than the original. \underline{\hspace{0.8in}}

\part The order of transformations doesn't matter (e.g., crop then rotate = rotate then crop). \underline{\hspace{0.8in}}
\end{parts}

\question[4] An image is 800×600 pixels. What will be its dimensions after:

\begin{parts}
\part Scaling by a factor of 0.5? \underline{\hspace{1.5in}}

\part Rotating 90° clockwise? \underline{\hspace{1.5in}}

\part Cropping to 300×200? \underline{\hspace{1.5in}}
\end{parts}

\end{questions}

\end{document}
