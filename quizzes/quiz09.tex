\documentclass{csexam}

\usepackage{listings}
\usepackage{xcolor}

% Python code styling
\lstset{
    language=Python,
    basicstyle=\ttfamily\small,
    keywordstyle=\color{blue}\bfseries,
    stringstyle=\color{red},
    commentstyle=\color{green!50!black}\itshape,
    showstringspaces=false,
    breaklines=true,
    frame=single,
    numbers=left,
    numberstyle=\tiny\color{gray}
}

% Course information
\course{CSCI 128: Introduction to Computer Science}
\courseshort{CSCI 128}
\exam{Quiz 9}
\examdate{Winter 2025}

\begin{document}

\maketitle

\begin{rules}
\textbf{Instructions:}
\begin{itemize}
    \item This quiz covers material from Lecture 12 (Animation and Course Review)
    \item Answer all questions in the space provided
    \item Show your work where applicable
    \item Total points: 15
    \item Time limit: 15 minutes
\end{itemize}
\end{rules}

\vspace{0.2in}

\showgrade

\newpage

\begin{questions}

\question[2] What is animation? How does it create the illusion of motion?
\vspace{1.5in}

\question[2] What does FPS stand for? If a video has 30 FPS, how many frames are displayed per second?
\vspace{1.5in}

\question[2] What is a frame in the context of animation?
\vspace{1.5in}

\question[3] To create a simple animation where a circle moves across the screen, you would generate multiple images. Describe the general approach for creating these frames.
\vspace{2in}

\question[3] Match each media type with its basic unit:
\begin{parts}
    \part Images are made of: \underline{\hspace{2in}}
    \part Sounds are made of: \underline{\hspace{2in}}
    \part Animations are made of: \underline{\hspace{2in}}
\end{parts}

\question[3] Throughout this course, we've learned to process images, sounds, and text. Give one example of how you might combine at least two of these media types in a single program.
\vspace{2in}

\end{questions}

\end{document}
