\documentclass{csexam}

\usepackage{listings}
\usepackage{xcolor}

% Python code styling
\lstset{
    language=Python,
    basicstyle=\ttfamily\small,
    keywordstyle=\color{blue}\bfseries,
    stringstyle=\color{red},
    commentstyle=\color{green!50!black}\itshape,
    showstringspaces=false,
    breaklines=true,
    frame=single,
    numbers=left,
    numberstyle=\tiny\color{gray}
}

% Course information
\course{CSCI 128: Coding for Problem Solving}
\courseshort{CSCI 128}
\exam{Quiz 4}
\examdate{Winter 2026}

\begin{document}

\maketitle

\begin{rules}
\textbf{Instructions:}
\begin{itemize}
    \item This quiz covers material from Lecture 4 (Picture Manipulation and Filters)
    \item Answer all questions in the space provided
    \item Show your work where applicable
    \item Time limit: 15 minutes
\end{itemize}
\end{rules}

\vspace{0.2in}

\showgrade

\newpage

\begin{questions}

\question[2] For an image with dimensions 640×480, how many total iterations will the nested loop perform?
\vspace{1in}

\question[3] What are the two methods for converting a color image to grayscale discussed in class? Give the formulas.
\begin{parts}
    \part \underline{\hspace{3in}}
    \part \underline{\hspace{3in}}
\end{parts}

\vspace{0.2cm}

\question[2] Why does the luminance method produce more natural-looking grayscale images?
\vspace{1.5in}

\question[2] What formula is used to create a negative image? If a pixel has RGB values (200, 50, 100), what would its values be after applying the negative filter?

Original: (200, 50, 100) \hspace{0.5in} Negative: \underline{\hspace{2in}}
\vspace{0.3cm}

\question[3] Given the following code, what does it do to the image?
\begin{lstlisting}
from PIL import Image
img = Image.open("photo.jpg")
pixels = img.load()
width, height = img.size

for x in range(width):
    for y in range(height):
        r, g, b = pixels[x, y]
        new_r = min(r + 75, 255)
        new_g = min(g + 75, 255)
        new_b = min(b + 75, 255)
        pixels[x, y] = (new_r, new_g, new_b)
\end{lstlisting}

\newpage

\question[3] Write Python code to apply a simple red tint to an image by reducing the blue and green channels by 50\%.

\vspace{0.5in}

\end{questions}

\end{document}
