\documentclass{csexam}

\usepackage{listings}
\usepackage{xcolor}

% Python code styling
\lstset{
    language=Python,
    basicstyle=\ttfamily\small,
    keywordstyle=\color{blue}\bfseries,
    stringstyle=\color{red},
    commentstyle=\color{green!50!black}\itshape,
    showstringspaces=false,
    breaklines=true,
    frame=single,
    numbers=left,
    numberstyle=\tiny\color{gray}
}

% Course information
\course{CSCI 128: Introduction to Computer Science}
\courseshort{CSCI 128}
\exam{Quiz 3}
\examdate{Fall 2024}

\begin{document}

\maketitle

\begin{rules}
\textbf{Instructions:}
\begin{itemize}
    \item This quiz covers material from Lectures 5-6 (Transformations and Collages)
    \item Answer all questions in the space provided
    \item Show your work where applicable
    \item Time limit: 15 minutes
\end{itemize}
\end{rules}

\vspace{0.2in}

\showgrade

\newpage

\begin{questions}

\question[2] What coordinate transformation is needed to mirror an image horizontally? If an image has width $W$ and a pixel is at position $x$, what is its new position after mirroring?
\vspace{1.5in}

\question[3] When rotating an image 90 degrees clockwise, what happens to the image dimensions?
\begin{parts}
    \part If the original image is 800×600, what are the dimensions after rotation?
    \part Why do the dimensions change?
\end{parts}
\vspace{1.5in}

\question[2] What is the difference between cropping and scaling an image?
\vspace{1.5in}

\question[3] In the context of image processing, what does it mean to "paste" one image onto another? What information do you need to perform this operation?
\vspace{1.5in}

\question[3] Given two images of the same size, explain how image blending works. What does the alpha parameter control?
\vspace{1.5in}

\question[2] Write Python code using Pillow to paste one image onto another at position (50, 50).
\begin{lstlisting}[numbers=none]


\end{lstlisting}
\vspace{0.5in}

\question[2] What is chromakey (green screen) compositing?
\vspace{1.5in}

\end{questions}

\end{document}
