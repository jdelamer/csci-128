\documentclass{csexam}

\usepackage{listings}
\usepackage{xcolor}

% Python code styling
\lstset{
    language=Python,
    basicstyle=\ttfamily\small,
    keywordstyle=\color{blue}\bfseries,
    stringstyle=\color{red},
    commentstyle=\color{green!50!black}\itshape,
    showstringspaces=false,
    breaklines=true,
    frame=single,
    numbers=left,
    numberstyle=\tiny\color{gray}
}

% Course information
\course{CSCI 128: Introduction to Computer Science}
\courseshort{CSCI 128}
\exam{Quiz 12: Animation and Course Review}
\examdate{Winter 2025}

\begin{document}

\maketitle

\begin{rules}
\textbf{Instructions:}
\begin{itemize}
    \item This quiz should take approximately 15 minutes.
    \item Answer all questions in the space provided.
    \item Show your work where applicable.
    \item No calculators or electronic devices allowed.
\end{itemize}
\end{rules}

\showgrade

\begin{questions}

\question[4] Short answer questions about animation:

\begin{parts}
\part What is a frame in animation?

\vspace{0.7in}

\part What does FPS stand for and what is a typical value for smooth video?

\vspace{0.7in}

\part How does animation create the illusion of motion?

\vspace{0.7in}
\end{parts}

\question[5] Complete the code to save a list of images as an animated GIF:

\begin{lstlisting}
from PIL import Image

frames = [Image.open(f"frame_{i}.jpg") for i in range(10)]

frames[0]._______(
    "animation.gif",
    save_all=_______,
    append_images=frames[_____:],
    duration=_______,  # milliseconds per frame
    loop=_______  # 0 = loop forever
)
\end{lstlisting}

\question[6] Match each media computation concept with its description:

\begin{parts}
\part Pixel \hspace{2in} \underline{\hspace{1.5in}}

\part Sample \hspace{1.9in} \underline{\hspace{1.5in}}

\part RGB \hspace{2.1in} \underline{\hspace{1.5in}}

\part Filter \hspace{2in} \underline{\hspace{1.5in}}

\part Transformation \hspace{1.1in} \underline{\hspace{1.5in}}

\part CSV \hspace{2.1in} \underline{\hspace{1.5in}}

\vspace{0.3in}

Options:
\begin{itemize}
\item A) Single point of sound data
\item B) Color model using three channels
\item C) Changes image structure (rotate, mirror)
\item D) Smallest unit of a digital image
\item E) Changes pixel colors (grayscale, sepia)
\item F) Tabular data format with commas
\end{itemize}
\end{parts}

\question[5] Name and briefly describe three Python libraries we used in this course:

\vspace{2.5in}

\question[5] Review concepts - True or False:

\begin{parts}
\part Variables in Python must be declared with a type. \underline{\hspace{1in}}

\part RGB values range from 0 to 255. \underline{\hspace{1in}}

\part Functions must always return a value. \underline{\hspace{1in}}

\part A pixel at (0, 0) is in the bottom-left corner. \underline{\hspace{1in}}

\part The \texttt{with} statement automatically closes files. \underline{\hspace{1in}}

\part Sound sampling rate is measured in Hertz (Hz). \underline{\hspace{1in}}

\part JSON can represent hierarchical data structures. \underline{\hspace{1in}}

\part Applying a mirror transformation twice returns the original. \underline{\hspace{1in}}

\part Rotating an image 90° changes its dimensions. \underline{\hspace{1in}}

\part CSV files must have a header row. \underline{\hspace{1in}}
\end{parts}

\end{questions}

\end{document}
