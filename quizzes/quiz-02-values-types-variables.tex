\documentclass{csexam}

\usepackage{listings}
\usepackage{xcolor}

% Python code styling
\lstset{
    language=Python,
    basicstyle=\ttfamily\small,
    keywordstyle=\color{blue}\bfseries,
    stringstyle=\color{red},
    commentstyle=\color{green!50!black}\itshape,
    showstringspaces=false,
    breaklines=true,
    frame=single,
    numbers=left,
    numberstyle=\tiny\color{gray}
}

% Course information
\course{CSCI 128: Introduction to Computer Science}
\courseshort{CSCI 128}
\exam{Quiz: Values, Types, and Variables}
\examdate{Winter 2026}

\begin{document}

\maketitle

\begin{rules}
\textbf{Instructions:}
\begin{itemize}
    \item This quiz should take approximately 15 minutes.
    \item Answer all questions in the space provided.
    \item Show your work where applicable.
    \item No calculators or electronic devices allowed.
\end{itemize}
\end{rules}

\showgrade
\printanswers

\newpage

\begin{questions}

\question[5] For each of the following Python expressions, write the value and its data type:

\begin{parts}
\part \texttt{5 + 3 * 2}

\begin{solution}
Value: 11, Type: int
\end{solution}

\vspace{0.25in}

\part \texttt{"Hello" + "World"}

\begin{solution}
Value: "HelloWorld", Type: str
\end{solution}

\vspace{0.25in}

\part \texttt{10 / 3}

\begin{solution}
Value: 3.333..., Type: float
\end{solution}

\vspace{0.25in}

\part \texttt{10 // 3}

\begin{solution}
Value: 3, Type: int
\end{solution}

\vspace{0.25in}

\part \texttt{7 \% 3}

\begin{solution}
Value: 1, Type: int
\end{solution}

\vspace{0.25in}
\end{parts}

\question[3] What is the output of the following code?

\begin{lstlisting}
x = 5
y = 10
x = y
y = 20
print(x)
\end{lstlisting}

\begin{solution}
The output is: 10

Explanation: x is assigned 5, then x is reassigned to the value of y (which is 10). When y is then changed to 20, x remains 10 because it holds the value, not a reference to y.
\end{solution}

\vspace{0.25in}

\question[4] Write a Python program that asks the user for their name and age, then prints a message saying "Hello [name], you are [age] years old."

\begin{solution}
\begin{lstlisting}
name = input("Enter your name: ")
age = input("Enter your age: ")
print("Hello " + name + ", you are " + age + " years old.")
\end{lstlisting}

Alternative solution:
\begin{lstlisting}
name = input("Enter your name: ")
age = input("Enter your age: ")
print(f"Hello {name}, you are {age} years old.")
\end{lstlisting}
\end{solution}

\newpage

\question[3] Identify and correct the error(s) in the following code:

\begin{lstlisting}
name = input("Enter your name: ")
age = input("Enter your age: ")
next_year_age = age + 1
print("Next year you will be " + next_year_age)
\end{lstlisting}

\begin{solution}
Errors:
\begin{itemize}
    \item \texttt{age} is a string (from input), so adding 1 to it causes a TypeError
    \item \texttt{next\_year\_age} needs to be converted to string for concatenation
\end{itemize}

Corrected code:
\begin{lstlisting}
name = input("Enter your name: ")
age = input("Enter your age: ")
next_year_age = int(age) + 1
print("Next year you will be " + str(next_year_age))
\end{lstlisting}
\end{solution}

\vspace{0.5in}

\question[5] Circle the valid Python variable names:

\begin{choices}
\choice \texttt{my\_variable}
\choice \texttt{2nd\_place}
\choice \texttt{first-name}
\choice \texttt{firstName}
\choice \texttt{\_count}
\choice \texttt{for}
\choice \texttt{total\_sum}
\choice \texttt{my variable}
\end{choices}

\begin{solution}
Valid variable names (should be circled):
\begin{itemize}
    \item \texttt{my\_variable} - Valid
    \item \texttt{firstName} - Valid
    \item \texttt{\_count} - Valid
    \item \texttt{total\_sum} - Valid
\end{itemize}

Invalid variable names:
\begin{itemize}
    \item \texttt{2nd\_place} - Invalid (starts with a number)
    \item \texttt{first-name} - Invalid (contains hyphen)
    \item \texttt{for} - Invalid (reserved keyword)
    \item \texttt{my variable} - Invalid (contains space)
\end{itemize}
\end{solution}

\end{questions}

\end{document}
