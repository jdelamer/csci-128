\documentclass{csexam}

\usepackage{listings}
\usepackage{xcolor}

% Python code styling
\lstset{
    language=Python,
    basicstyle=\ttfamily\small,
    keywordstyle=\color{blue}\bfseries,
    stringstyle=\color{red},
    commentstyle=\color{green!50!black}\itshape,
    showstringspaces=false,
    breaklines=true,
    frame=single,
    numbers=left,
    numberstyle=\tiny\color{gray}
}

% Course information
\course{CSCI 128: Introduction to Computer Science}
\courseshort{CSCI 128}
\exam{Quiz 10: Text Manipulation and Files}
\examdate{Winter 2025}

\begin{document}

\maketitle

\begin{rules}
\textbf{Instructions:}
\begin{itemize}
    \item This quiz should take approximately 15 minutes.
    \item Answer all questions in the space provided.
    \item Show your work where applicable.
    \item No calculators or electronic devices allowed.
\end{itemize}
\end{rules}

\showgrade

\begin{questions}

\question[5] Complete the code to read all lines from a text file and print each line:

\begin{lstlisting}
with open("data.txt", "____") as file:
    for _____ in file:
        print(_____.strip())
\end{lstlisting}

\question[4] Write code to write three lines of text to a file called "output.txt":

\vspace{2in}

\question[4] What is the difference between opening a file with mode "w" versus mode "a"?

\vspace{1.3in}

\question[5] Given this text file content:

\begin{verbatim}
Alice,85,92,78
Bob,90,88,95
Carol,78,85,82
\end{verbatim}

Write code to read the file and calculate the average score for each student:

\vspace{2.5in}

\question[3] What does the \texttt{strip()} method do when reading lines from a file? Why is it useful?

\vspace{1.2in}

\question[4] Complete the code to count the number of words in a text file:

\begin{lstlisting}
with open("essay.txt", "r") as file:
    content = file.______()
    words = content.______()
    count = len(______)
    print(f"Word count: {count}")
\end{lstlisting}

\question[5] Short answer questions about file handling:

\begin{parts}
\part What is the advantage of using \texttt{with open()} instead of just \texttt{open()}?

\vspace{0.8in}

\part What happens if you try to open a file that doesn't exist in read mode?

\vspace{0.8in}

\part Why would you use \texttt{readlines()} instead of \texttt{read()}?

\vspace{0.8in}
\end{parts}

\end{questions}

\end{document}
