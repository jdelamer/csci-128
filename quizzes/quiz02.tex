\documentclass{csexam}

\usepackage{listings}
\usepackage{xcolor}

% Python code styling
\lstset{
    language=Python,
    basicstyle=\ttfamily\small,
    keywordstyle=\color{blue}\bfseries,
    stringstyle=\color{red},
    commentstyle=\color{green!50!black}\itshape,
    showstringspaces=false,
    breaklines=true,
    frame=single,
    numbers=left,
    numberstyle=\tiny\color{gray}
}

% Course information
\course{CSCI 128: Introduction to Computer Science}
\courseshort{CSCI 128}
\exam{Quiz 2}
\examdate{Winter 2025}

\begin{document}

\maketitle

\begin{rules}
\textbf{Instructions:}
\begin{itemize}
    \item This quiz covers material from Lecture 3 (Pictures and Pixels)
    \item Answer all questions in the space provided
    \item Show your work where applicable
    \item Time limit: 15 minutes
\end{itemize}
\end{rules}

\vspace{0.2in}

\showgrade

\newpage

\begin{questions}

\question[1] What does RGB stand for, and what is the range of values for each color channel?
\vspace{1.5in}

\question[3] What RGB values would create the following colors?
\begin{parts}
    \part Pure red: \underline{\hspace{2in}}
    \part White: \underline{\hspace{2in}}
    \part Black: \underline{\hspace{2in}}
\end{parts}

\question[2] In digital images, what is a pixel?
\vspace{1.5in}

\question[3] What is the coordinate of the top-left pixel in an image? Explain the coordinate system used for images.
\vspace{1.5in}

\question[3] Given the following code, what does it do?
\begin{lstlisting}
from PIL import Image
img = Image.open("photo.jpg")
r, g, b =  img.getpixel((100, 50))
print(r, g, b)
\end{lstlisting}
\vspace{2in}

\question[3] Write Python code using Pillow to create a new 100×100 white image.
\vspace{1.5in}



\end{questions}

\end{document}
