\documentclass{csexam}

\usepackage{listings}
\usepackage{xcolor}

% Python code styling
\lstset{
    language=Python,
    basicstyle=\ttfamily\small,
    keywordstyle=\color{blue}\bfseries,
    stringstyle=\color{red},
    commentstyle=\color{green!50!black}\itshape,
    showstringspaces=false,
    breaklines=true,
    frame=single,
    numbers=left,
    numberstyle=\tiny\color{gray}
}

% Course information
\course{CSCI 128: Introduction to Computer Science}
\courseshort{CSCI 128}
\exam{Quiz 7}
\examdate{Winter 2025}

\begin{document}

\maketitle

\begin{rules}
\textbf{Instructions:}
\begin{itemize}
    \item This quiz covers material from Lecture 10 (Text Manipulation and Files)
    \item Answer all questions in the space provided
    \item Show your work where applicable
    \item Time limit: 15 minutes
\end{itemize}
\end{rules}

\vspace{0.2in}

\showgrade

\newpage

\begin{questions}

\question[2] What is the difference between \texttt{open(filename, "r")} and \texttt{open(filename, "w")} in Python?
\vspace{1.5in}

\question[3] Write Python code to read all lines from a file called \texttt{data.txt} and store them in a list.
\vspace{1.5in}

\question[2] What does the \texttt{strip()} method do when applied to a string? Why is it useful when reading lines from a file?
\vspace{1.5in}

\question[2] What is the result of the following code?
\begin{lstlisting}[numbers=none]
text = "Hello,World,Python"
parts = text.split(",")
print(len(parts))
\end{lstlisting}
Answer: \underline{\hspace{1in}}

\question[3] Write Python code to write three names to a file called \texttt{names.txt}, each on a separate line.
\vspace{1.5in}

\question[3] What will the following code print?
\begin{lstlisting}[numbers=none]
filename = "report_2024.txt"
parts = filename.split(".")
print(parts[0])
\end{lstlisting}
Answer: \underline{\hspace{2in}}

\end{questions}

\end{document}
