\documentclass{csexam}

\usepackage{listings}
\usepackage{xcolor}

% Python code styling
\lstset{
    language=Python,
    basicstyle=\ttfamily\small,
    keywordstyle=\color{blue}\bfseries,
    stringstyle=\color{red},
    commentstyle=\color{green!50!black}\itshape,
    showstringspaces=false,
    breaklines=true,
    frame=single,
    numbers=left,
    numberstyle=\tiny\color{gray}
}

% Course information
\course{CSCI 128: Introduction to Computer Science}
\courseshort{CSCI 128}
\exam{Quiz 3: Pictures and Pixels}
\examdate{Winter 2025}

\begin{document}

\maketitle

\begin{rules}
\textbf{Instructions:}
\begin{itemize}
    \item This quiz should take approximately 15 minutes.
    \item Answer all questions in the space provided.
    \item Show your work where applicable.
    \item No calculators or electronic devices allowed.
\end{itemize}
\end{rules}

\showgrade

\begin{questions}

\question[4] Short answer questions about digital images:

\begin{parts}
\part What does RGB stand for and what does each value represent?

\vspace{0.8in}

\part What is the range of valid values for each color channel in the RGB color model?

\vspace{0.5in}

\part What color is represented by RGB(255, 0, 0)?

\vspace{0.3in}

\part What color is represented by RGB(0, 0, 0)?

\vspace{0.3in}
\end{parts}

\question[4] Fill in the blanks to complete the code that loads an image and gets its dimensions:

\begin{lstlisting}
from PIL import _______

img = Image.____("photo.jpg")
width, height = img._____
pixels = img._____()
\end{lstlisting}

\vspace{0.5in}

\question[5] Write code to access the pixel at position (10, 20) in an image, get its RGB values, and change it to pure blue (0, 0, 255).

\vspace{2in}

\question[4] What will be the color of the pixel after this code executes?

\begin{lstlisting}
pixels[50, 75] = (100, 150, 200)
r, g, b = pixels[50, 75]
pixels[50, 75] = (r + 50, g - 50, b + 55)
\end{lstlisting}

Answer: RGB(\underline{\hspace{0.5in}}, \underline{\hspace{0.5in}}, \underline{\hspace{0.5in}})

\vspace{0.3in}

\question[3] Write nested loops to process every pixel in an image with width=800 and height=600:

\vspace{2in}

\end{questions}

\end{document}
