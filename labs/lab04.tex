\documentclass[11pt]{article}

\usepackage[margin=1in]{geometry}
\usepackage[most]{tcolorbox}
\usepackage{fancyhdr}
\usepackage{graphicx}
\usepackage{listings}
\usepackage{xcolor}
\usepackage{hyperref}
\usepackage{amsmath}
\usepackage{enumitem}

% Python code styling
\lstset{
    language=Python,
    basicstyle=\ttfamily\small,
    keywordstyle=\color{blue}\bfseries,
    stringstyle=\color{red},
    commentstyle=\color{green!50!black}\itshape,
    columns=fullflexible,
    showstringspaces=false,
    breaklines=true,
    frame=single,
    numbers=left,
    numberstyle=\tiny\color{gray},
    tabsize=4
}

% Custom environments
\newtcolorbox{objectives}{
    colback=blue!5!white,
    colframe=blue!75!black,
    title=Learning Objectives,
    fonttitle=\bfseries
}

\newtcolorbox{grading}{
    colback=green!5!white,
    colframe=green!50!black,
    title=Grading,
    fonttitle=\bfseries
}

\newtcolorbox{setup}{
    colback=orange!5!white,
    colframe=orange!75!black,
    title=Setup,
    fonttitle=\bfseries
}

\newtcolorbox{note}{
    colback=yellow!5!white,
    colframe=yellow!75!black,
    title=Note,
    fonttitle=\bfseries
}

\newtcolorbox{hint}{
    colback=purple!5!white,
    colframe=purple!75!black,
    title=Hint,
    fonttitle=\bfseries
}

\newcounter{taskcounter}
\newcommand{\task}[1]{%
    \stepcounter{taskcounter}%
    \subsection*{Task \thetaskcounter: #1}%
}

% Header and footer
\pagestyle{fancy}
\fancyhf{}
\lhead{CSCI 128: Coding for Problem Solving}
\rhead{Lab 04}
\cfoot{\thepage}

\title{Lab 04: Functions and Image Transformations}
\author{CSCI 128: Coding for Problem Solving}
\date{}

\begin{document}

\maketitle

\begin{grading}
\textbf{Participation-Based Grading:} Complete the tasks during the lab session and get checked off by the instructor or TA.

\vspace{0.5em}

\textbf{To receive credit:}
\begin{itemize}
    \item Attend the entire lab session
    \item Complete Tasks 1-4
    \item Submit your work on Moodle
\end{itemize}
\end{grading}

\section{Introduction}

Today you'll learn \textbf{functions} (reusable code blocks) and apply them to \textbf{image transformations} (mirror, rotate, crop, scale). Unlike previous labs where we changed pixel colors, today we change pixel positions.

\section{Setup}

\begin{setup}
\textbf{Before you begin:}
\begin{enumerate}
    \item Create a folder: \texttt{csci128/lab04}
    \item Ensure Pillow is installed: \texttt{pip install Pillow}
    \item Download several test images (landscapes and portraits work well)
    \item Put images in an \texttt{images} subfolder inside \texttt{lab04}
\end{enumerate}

\textbf{Recommended test images:}
\begin{itemize}
    \item A square image (same width and height)
    \item A rectangular landscape image (wider than tall)
    \item A rectangular portrait image (taller than wide)
    \item An image with recognizable text or asymmetric features (to clearly see transformations)
\end{itemize}
\end{setup}

\section{Lab Tasks}

\task{Functions and Transformations Warmup}

\textbf{Create \texttt{warmup.py} - Write these simple functions:}

\begin{lstlisting}
# TODO 1: Function that doubles a number
def double(x):
    # Your code: return x * 2
    pass

# TODO 2: Function that averages two numbers
def average(a, b):
    # Your code: return (a + b) / 2
    pass

# Test them
print(double(5))        # Should print 10
print(average(10, 20))  # Should print 15.0
\end{lstlisting}

\textbf{Function syntax reminder:}
\begin{itemize}
    \item \texttt{def function\_name(parameters):}
    \item Indented code block
    \item \texttt{return result}
\end{itemize}

\task{Mirror and Rotate Functions}

\textbf{Create \texttt{transforms.py}:}

\begin{lstlisting}
from PIL import Image

# ===== MIRROR FUNCTIONS =====
def mirror_horizontal(img):
    """Mirror image horizontally (flip left-right)"""
    width, height = img.size
    mirrored = Image.new("RGB", (width, height))
    src_pixels = img.load()
    dst_pixels = mirrored.load()

    # TODO: Mirror pixels - use formula: new_x = width - 1 - x
    for x in range(width):
        for y in range(height):
            # Your code here
            pass
    return mirrored

def mirror_vertical(img):
    """Mirror image vertically (flip top-bottom)"""
    # TODO: Write this function - use formula: new_y = height - 1 - y
    pass

# ===== ROTATION FUNCTIONS =====
def rotate_90_cw(img):
    """Rotate 90 degrees clockwise - dimensions swap!"""
    width, height = img.size
    rotated = Image.new("RGB", (height, width))  # Swapped!
    src_pixels = img.load()
    dst_pixels = rotated.load()

    # TODO: Use formulas: new_x = height - 1 - y, new_y = x
    for x in range(width):
        for y in range(height):
            # Your code here
            pass
    return rotated

def rotate_180(img):
    """Rotate 180 degrees - dimensions stay same"""
    # TODO: Write this - formulas: new_x = width - 1 - x, new_y = height - 1 - y
    pass

# Test your functions
img = Image.open("images/sample.jpg")
mirror_horizontal(img).save("mirror_h.jpg")
mirror_vertical(img).save("mirror_v.jpg")
rotate_90_cw(img).save("rotate_90.jpg")
rotate_180(img).save("rotate_180.jpg")
print("Created mirror and rotation images!")
\end{lstlisting}

\textbf{Transformation formulas:}
\begin{itemize}
    \item Horizontal: $(x, y) \rightarrow (W-1-x, y)$
    \item Vertical: $(x, y) \rightarrow (x, H-1-y)$
    \item 90$^\circ$ CW: $(x, y) \rightarrow (H-1-y, x)$ [dims swap to $(H, W)$]
    \item 180$^\circ$: $(x, y) \rightarrow (W-1-x, H-1-y)$
\end{itemize}

\task{Crop and Scale Functions}

\begin{lstlisting}
from PIL import Image

def crop_image(img, start_x, start_y, crop_width, crop_height):
    """Crop a rectangular region"""
    cropped = Image.new("RGB", (crop_width, crop_height))
    src_pixels = img.load()
    dst_pixels = cropped.load()

    # TODO: Copy pixels from (start_x+x, start_y+y) to (x, y)
    for x in range(crop_width):
        for y in range(crop_height):
            # Your code here
            pass
    return cropped

def crop_center(img, crop_width, crop_height):
    """Crop center region"""
    # TODO: Calculate start positions, then use crop_image()
    # Hint: start_x = (width - crop_width) // 2
    pass

# ===== SCALE FUNCTIONS =====
def scale_by_factor(img, factor):
    """Scale image by factor"""
    width, height = img.size
    # TODO: Calculate new dimensions and use img.resize()
    pass

def create_thumbnail(img, max_size):
    """Create thumbnail within max_size"""
    # TODO: Find larger dimension, calculate factor, use scale_by_factor()
    pass

# Test your functions
crop_image(img, 100, 100, 400, 300).save("crop.jpg")
crop_center(img, 300, 300).save("center.jpg")
scale_by_factor(img, 0.5).save("half.jpg")
create_thumbnail(img, 200).save("thumb.jpg")
print("Created crop and scale images!")
\end{lstlisting}

\task{Combining Transformations}

Now combine your functions!

\textbf{Create \texttt{combine.py}:}
\begin{lstlisting}
from PIL import Image
from transforms import *

img = Image.open("images/sample.jpg")

# TODO 1: Create rotation sequence (0, 90, 180 degrees)
# Save as panel_0.jpg, panel_90.jpg, panel_180.jpg

# TODO 2: Crop center 400x400, then mirror horizontal
# Save as center_mirror.jpg

# TODO 3: Create thumbnails at sizes 100, 200, 300
for size in [100, 200, 300]:
    # Your code here
    pass

# TODO 4: Combine transforms - rotate then mirror
combined = # mirror_horizontal(rotate_90_cw(img))
combined.save("combined.jpg")

print("Done!")
\end{lstlisting}

\textbf{Remember:} Order matters! Crop→Rotate $\neq$ Rotate→Crop

\section{Bonus Challenges}

\subsection*{Challenge 1: Mirror Both}
Write \texttt{mirror\_both(img)} that mirrors horizontally AND vertically.

\subsection*{Challenge 2: Batch Processing}
Apply same transformation to all images in a folder.

\subsection*{Challenge 3: Transformation Pipeline}
\begin{lstlisting}
def apply_pipeline(img, transforms):
    """Apply list of transforms: ['rotate_90_cw', 'mirror_horizontal']"""
    pass
\end{lstlisting}

\section{Common Issues}

\textbf{IndexError: image index out of range}
\begin{itemize}
    \item Check your coordinate calculations
    \item Ensure you're not accessing pixels outside bounds
    \item Remember: coordinates go from 0 to (width-1) and 0 to (height-1)
    \item Check that you swapped dimensions for 90/270 rotations
\end{itemize}

\textbf{Function returns None}
\begin{itemize}
    \item Make sure you have a \texttt{return} statement
    \item Check that the return statement is properly indented
    \item Verify you're returning the modified image, not the original
\end{itemize}

\textbf{Rotated image looks wrong}
\begin{itemize}
    \item Verify you swapped dimensions for 90/270 rotations
    \item Double-check your transformation formulas
    \item Test with an asymmetric image (e.g., with text) to see rotation clearly
\end{itemize}

\textbf{NameError when calling a function}
\begin{itemize}
    \item Make sure the function is defined before you call it
    \item Check for typos in the function name
    \item If importing, ensure the file is in the same directory or use proper import path
\end{itemize}


\end{document}
