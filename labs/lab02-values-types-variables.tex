\documentclass[11pt]{article}

\usepackage[margin=1in]{geometry}
\usepackage{fancyhdr}
\usepackage{graphicx}
\usepackage{listings}
\usepackage{xcolor}
\usepackage{hyperref}
\usepackage{amsmath}

% Python code styling
\lstset{
    language=Python,
    basicstyle=\ttfamily\small,
    keywordstyle=\color{blue}\bfseries,
    stringstyle=\color{red},
    commentstyle=\color{green!50!black}\itshape,
    showstringspaces=false,
    breaklines=true,
    frame=single,
    numbers=left,
    numberstyle=\tiny\color{gray}
}

% Header and footer
\pagestyle{fancy}
\fancyhf{}
\lhead{CSCI 128: Introduction to Computer Science}
\rhead{Lab 02}
\cfoot{\thepage}

\begin{document}

\begin{center}
    {\LARGE \textbf{Lab 02: Values, Types, and Variables}}\\
    \vspace{0.5em}
    {\large Exploring Python's Building Blocks}\\
    \vspace{1em}
    \includegraphics[width=0.3\textwidth]{LogoWordsBottom.png}
\end{center}

\section*{Introduction}

Welcome to your second lab! Today you'll explore the fundamental building blocks of Python programs: values, types, and variables. You'll learn how Python stores and manipulates different kinds of data, and practice using variables to make your programs more powerful and flexible.

\section*{Getting Started}

Open your Python environment (IDLE, Thonny, or your preferred IDE) and create a new file called \texttt{lab02.py}.

\section{Exploring Data Types}

Python has several built-in data types. Let's explore them!

\subsection*{Activity 1: Type Detective}

Write code to determine the type of each of these values. Use the \texttt{type()} function to investigate:

\begin{lstlisting}
# Example
print(type(42))  # What type is this?

# Now try these:
# 1. 3.14159
# 2. "Hello, World!"
# 3. True
# 4. 100
# 5. "42"
# 6. 0.0
\end{lstlisting}

\textbf{Question:} What's the difference between \texttt{42} and \texttt{"42"}? Try adding them together and see what happens!

\subsection*{Activity 2: Type Conversion}

Sometimes you need to convert between types. Experiment with these conversions:

\begin{lstlisting}
# Converting strings to numbers
age_str = "25"
age_num = int(age_str)
print("Age as number:", age_num)

# Converting numbers to strings
score = 98.5
score_str = str(score)
print("Score as string:", score_str)

# Try this: What happens when you convert 3.7 to an int?
\end{lstlisting}

\textbf{Challenge:} Ask the user for two numbers (as strings) using \texttt{input()}, convert them to integers, and print their sum.

\section{Working with Variables}

Variables are like labeled boxes that store values. They make your programs flexible and readable!

\subsection*{Activity 3: Variable Playground}

Create variables to store information about yourself:

\begin{lstlisting}
# Personal information
name = "Your Name"
age = 20
height_inches = 68
is_student = True
favorite_number = 7.5

# Print them all
print("Name:", name)
print("Age:", age)
# Continue for the rest...
\end{lstlisting}

Now try updating variables:

\begin{lstlisting}
# Variables can change!
score = 0
print("Starting score:", score)

score = score + 10
print("After bonus:", score)

score = score * 2
print("After doubling:", score)
\end{lstlisting}

\subsection*{Activity 4: Variable Naming Challenge}

Which of these are valid variable names in Python? Try them out and see!

\begin{lstlisting}
# Valid or not?
# my_variable = 5
# 2fast = 10
# fast2 = 10
# user-name = "Alice"
# user_name = "Alice"
# class = "CSCI 128"
# my_class = "CSCI 128"
\end{lstlisting}

\textbf{Remember:} Variable names should be descriptive! Use \texttt{student\_count} instead of \texttt{sc}.

\section{Fun with Math and Strings}

\subsection*{Activity 5: Calculator}

Create a simple calculator using variables:

\begin{lstlisting}
# Temperature converter
fahrenheit = 72
celsius = (fahrenheit - 32) * 5 / 9
print(f"{fahrenheit}F is {celsius}C")

# Circle calculator
radius = 5
pi = 3.14159
area = pi * radius ** 2
circumference = 2 * pi * radius
print(f"Circle with radius {radius}:")
print(f"  Area: {area}")
print(f"  Circumference: {circumference}")
\end{lstlisting}

\textbf{Your Turn:} Create a rectangle calculator that computes area and perimeter given length and width.

\subsection*{Activity 6: String Magic}

Strings have special powers! Explore string operations:

\begin{lstlisting}
# String concatenation
first_name = "Ada"
last_name = "Lovelace"
full_name = first_name + " " + last_name
print(full_name)

# String repetition
cheer = "Go! "
print(cheer * 5)

# String formatting
name = "Python"
version = 3.11
print(f"I love {name} {version}!")
\end{lstlisting}

\textbf{Challenge:} Ask the user for their name and favorite color, then print a personalized message like: ``Hello, Alice! Blue is a wonderful color!''

\section{Interactive Programs}

\subsection*{Activity 7: User Input}

Make your programs interactive by getting input from users:

\begin{lstlisting}
# Simple greeting
name = input("What's your name? ")
print(f"Hello, {name}! Nice to meet you!")

# Personalized fortune teller
print("\nFortune Teller 3000")
lucky_number = input("What's your lucky number? ")
favorite_animal = input("What's your favorite animal? ")
print(f"\nYour fortune: A {favorite_animal} will bring you {lucky_number} good things today!")
\end{lstlisting}

\textbf{Your Turn:} Create a ``Mad Libs'' style program that asks for several inputs (noun, verb, adjective, etc.) and uses them to create a funny story!

\section{Putting It All Together}

\subsection*{Final Challenge: Personal Profile Generator}

Create a program that:
\begin{enumerate}
    \item Asks the user for their name, age, favorite hobby, and dream vacation destination
    \item Calculates how old they'll be in 10 years
    \item Creates a nicely formatted profile using all this information
    \item Uses both variables and string formatting
\end{enumerate}

\textbf{Example Output:}
\begin{verbatim}
=== PERSONAL PROFILE ===
Name: Alex Johnson
Current Age: 20
Age in 10 years: 30
Favorite Hobby: Photography
Dream Destination: Tokyo

Fun Fact: In 10 years, you'll be traveling to Tokyo!
\end{verbatim}

\section*{Bonus Explorations}

If you finish early, try these fun challenges:

\begin{itemize}
    \item Create a ``this or that'' quiz that asks users to choose between options
    \item Build a simple tip calculator that computes 15\%, 18\%, and 20\% tips
    \item Make a acronym generator that takes words and creates an acronym from their first letters
    \item Design a ``year born'' calculator that figures out birth year from age
\end{itemize}

\section*{Reflection Questions}

Think about these questions as you work:
\begin{itemize}
    \item Why is it important to use descriptive variable names?
    \item When might you need to convert between types?
    \item How do variables make programs more flexible than just using literal values?
    \item What happens if you try to use a variable before assigning it a value?
\end{itemize}

\vspace{2em}

\begin{center}
\textit{Have fun exploring Python's building blocks! \\
Every great program starts with these fundamentals.}
\end{center}

\end{document}
