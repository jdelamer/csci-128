\documentclass{cslab}

\usepackage{listings}
\usepackage{xcolor}

% Python code styling
\lstset{
    language=Python,
    basicstyle=\ttfamily\small,
    keywordstyle=\color{blue}\bfseries,
    stringstyle=\color{red},
    commentstyle=\color{green!50!black}\itshape,
    showstringspaces=false,
    breaklines=true,
    frame=single,
    numbers=left,
    numberstyle=\tiny\color{gray},
    tabsize=4
}

% Lab information
\labnumber{10}
\labtitle{Web Data and Structured Data}
\course{CSCI 128: Introduction to Computer Science}
\courseshort{CSCI 128}
\semester{Fall 2024}
\timelimit{2 hours}
\logoimage{LogoWordsBottom.png}

\begin{document}

\maketitle

\begin{objectives}
    \item Work with CSV data files
    \item Parse and generate JSON data
    \item Process structured data from multiple sources
    \item Create data-driven image processing applications
    \item Build image galleries from metadata
    \item Generate reports combining data and images
    \item Understand data serialization formats
\end{objectives}

\begin{grading}
\textbf{Participation-Based Grading:} Complete the tasks during the lab session and get checked off by the instructor or TA.

\textbf{To receive credit:}
\begin{itemize}
    \item Attend the entire lab session
    \item Complete Tasks 1-6 (Bonus is optional)
    \item Demonstrate your data processing programs
    \item Get checked off before leaving
\end{itemize}
\end{grading}

\section{Introduction}

Real-world applications often combine data from multiple sources with media processing. Today you'll learn to work with structured data formats like CSV and JSON, process metadata, and create data-driven applications. You'll build programs that read processing instructions from files, generate image galleries, and create automated workflows!

\section{Setup}

\begin{setup}
\textbf{Before you begin:}
\begin{enumerate}
    \item Create a folder: \texttt{csci128/lab10}
    \item Create subfolders: \texttt{data}, \texttt{images}, \texttt{output}, \texttt{gallery}
    \item Ensure Pillow is installed: \texttt{pip install Pillow}
    \item Have 5-10 sample images ready
\end{enumerate}
\end{setup}

\section{Lab Tasks}

\task{Advanced CSV Processing}

Work with CSV data for image processing tasks.

\textbf{Create \texttt{data/image\_tasks.csv}:}
\begin{verbatim}
filename,operation,parameter,output_name
photo1.jpg,grayscale,,photo1_gray.jpg
photo2.jpg,brightness,50,photo2_bright.jpg
photo3.jpg,resize,400x300,photo3_small.jpg
photo1.jpg,sepia,,photo1_sepia.jpg
photo2.jpg,rotate,90,photo2_rotated.jpg
\end{verbatim}

\textbf{Create \texttt{csv\_batch\_processor.py}:}
\begin{lstlisting}
"""
CSV-driven batch image processor
"""
from PIL import Image
import csv
import os

def grayscale(img):
    """Convert to grayscale."""
    width, height = img.size
    pixels = img.load()
    result = Image.new("RGB", (width, height))
    result_pixels = result.load()

    for y in range(height):
        for x in range(width):
            r, g, b = pixels[x, y]
            gray = int(0.299 * r + 0.587 * g + 0.114 * b)
            result_pixels[x, y] = (gray, gray, gray)

    return result

def adjust_brightness(img, amount):
    """Adjust brightness."""
    width, height = img.size
    pixels = img.load()
    result = Image.new("RGB", (width, height))
    result_pixels = result.load()

    amount = int(amount)
    for y in range(height):
        for x in range(width):
            r, g, b = pixels[x, y]
            result_pixels[x, y] = (
                max(0, min(255, r + amount)),
                max(0, min(255, g + amount)),
                max(0, min(255, b + amount))
            )

    return result

def sepia(img):
    """Apply sepia tone."""
    width, height = img.size
    pixels = img.load()
    result = Image.new("RGB", (width, height))
    result_pixels = result.load()

    for y in range(height):
        for x in range(width):
            r, g, b = pixels[x, y]
            new_r = int(0.393 * r + 0.769 * g + 0.189 * b)
            new_g = int(0.349 * r + 0.686 * g + 0.168 * b)
            new_b = int(0.272 * r + 0.534 * g + 0.131 * b)
            result_pixels[x, y] = (
                min(255, new_r),
                min(255, new_g),
                min(255, new_b)
            )

    return result

def resize_image(img, size_str):
    """Resize image based on 'WIDTHxHEIGHT' string."""
    width, height = map(int, size_str.split('x'))
    return img.resize((width, height))

def rotate_image(img, angle):
    """Rotate image by angle."""
    angle = int(angle)
    if angle == 90:
        return img.transpose(Image.ROTATE_270)
    elif angle == 180:
        return img.transpose(Image.ROTATE_180)
    elif angle == 270:
        return img.transpose(Image.ROTATE_90)
    return img

def process_csv_tasks(csv_file):
    """Process image tasks from CSV file."""
    operations = {
        'grayscale': grayscale,
        'brightness': adjust_brightness,
        'sepia': sepia,
        'resize': resize_image,
        'rotate': rotate_image
    }

    with open(csv_file, 'r') as file:
        reader = csv.DictReader(file)

        for row in reader:
            filename = row['filename']
            operation = row['operation']
            parameter = row['parameter']
            output_name = row['output_name']

            input_path = os.path.join('images', filename)
            output_path = os.path.join('output', output_name)

            try:
                img = Image.open(input_path)

                # Apply operation
                if operation in operations:
                    if parameter:
                        result = operations[operation](img, parameter)
                    else:
                        result = operations[operation](img)

                    result.save(output_path)
                    print(f"[OK] {filename} -> {operation} -> {output_name}")
                else:
                    print(f"[X] Unknown operation: {operation}")

            except Exception as e:
                print(f"[X] Error processing {filename}: {e}")

# Run the processor
print("Processing tasks from CSV...")
process_csv_tasks('data/image_tasks.csv')
print("\nDone!")
\end{lstlisting}

\textbf{Experiments:}
\begin{enumerate}
    \item Add more operations to the CSV
    \item Create a CSV with 20+ processing tasks
    \item Add a new operation (negative, mirror, etc.)
    \item Process the same image with multiple filters
\end{enumerate}

\task{Working with JSON Data}

Use JSON for more complex data structures.

\textbf{Create \texttt{json\_processor.py}:}
\begin{lstlisting}
"""
JSON data processing for images
"""
import json
from PIL import Image
import os

# Create sample JSON data
sample_data = {
    "project_name": "Photo Collection 2024",
    "author": "John Doe",
    "images": [
        {
            "filename": "photo1.jpg",
            "title": "Beautiful Sunset",
            "location": "California",
            "date": "2024-01-15",
            "tags": ["nature", "sunset", "beach"],
            "filters": ["grayscale", "brightness:30"]
        },
        {
            "filename": "photo2.jpg",
            "title": "Mountain View",
            "location": "Colorado",
            "date": "2024-02-20",
            "tags": ["nature", "mountain", "snow"],
            "filters": ["sepia"]
        },
        {
            "filename": "photo3.jpg",
            "title": "City Lights",
            "location": "New York",
            "date": "2024-03-10",
            "tags": ["urban", "night", "city"],
            "filters": ["brightness:20"]
        }
    ],
    "output_settings": {
        "format": "jpg",
        "quality": 90,
        "add_watermark": True,
        "watermark_text": "(c) 2024 John Doe"
    }
}

# Save sample JSON
with open('data/photo_project.json', 'w') as f:
    json.dump(sample_data, f, indent=2)

print("Created data/photo_project.json")

# Read and parse JSON
with open('data/photo_project.json', 'r') as f:
    data = json.load(f)

print(f"\nProject: {data['project_name']}")
print(f"Author: {data['author']}")
print(f"Number of images: {len(data['images'])}")

# Display image information
print("\nImages in project:")
for img_data in data['images']:
    print(f"\n  {img_data['filename']}")
    print(f"    Title: {img_data['title']}")
    print(f"    Location: {img_data['location']}")
    print(f"    Date: {img_data['date']}")
    print(f"    Tags: {', '.join(img_data['tags'])}")
    print(f"    Filters: {', '.join(img_data['filters'])}")

# Search by tag
search_tag = "nature"
print(f"\n\nImages tagged with '{search_tag}':")
for img_data in data['images']:
    if search_tag in img_data['tags']:
        print(f"  - {img_data['title']} ({img_data['filename']})")

# Search by location
search_location = "California"
print(f"\nImages from {search_location}:")
for img_data in data['images']:
    if img_data['location'] == search_location:
        print(f"  - {img_data['title']}")

# Create report
report_file = 'output/json_report.txt'
with open(report_file, 'w') as f:
    f.write("="*60 + "\n")
    f.write(f"PROJECT REPORT: {data['project_name']}\n")
    f.write("="*60 + "\n\n")
    f.write(f"Author: {data['author']}\n")
    f.write(f"Total images: {len(data['images'])}\n\n")

    # Count tags
    all_tags = []
    for img_data in data['images']:
        all_tags.extend(img_data['tags'])

    unique_tags = set(all_tags)
    f.write("Tags used:\n")
    for tag in sorted(unique_tags):
        count = all_tags.count(tag)
        f.write(f"  {tag}: {count} images\n")

    f.write("\n" + "="*60 + "\n")

print(f"\nReport saved to {report_file}")
\end{lstlisting}

\textbf{JSON experiments:}
\begin{enumerate}
    \item Add more images to the JSON file
    \item Add new metadata fields (camera, ISO, aperture)
    \item Create JSON for different project types
    \item Export image metadata to JSON
\end{enumerate}

\task{Building an Image Gallery}

Create an HTML gallery from image metadata.

\textbf{Create \texttt{gallery\_generator.py}:}
\begin{lstlisting}
"""
Generate HTML image gallery from data
"""
import json
from PIL import Image
import os

def create_thumbnail(img, size=200):
    """Create thumbnail maintaining aspect ratio."""
    img.thumbnail((size, size))
    return img

def generate_gallery(json_file, output_dir='gallery'):
    """Generate HTML gallery from JSON metadata."""

    # Load data
    with open(json_file, 'r') as f:
        data = json.load(f)

    # Create output directory
    if not os.path.exists(output_dir):
        os.makedirs(output_dir)

    thumb_dir = os.path.join(output_dir, 'thumbnails')
    if not os.path.exists(thumb_dir):
        os.makedirs(thumb_dir)

    # Generate thumbnails
    print("Generating thumbnails...")
    for img_data in data['images']:
        filename = img_data['filename']
        input_path = os.path.join('images', filename)

        try:
            img = Image.open(input_path)
            thumb = img.copy()
            create_thumbnail(thumb)

            thumb_path = os.path.join(thumb_dir, filename)
            thumb.save(thumb_path)
            print(f"  [OK] {filename}")
        except Exception as e:
            print(f"  [X] {filename}: {e}")

    # Generate HTML
    html_file = os.path.join(output_dir, 'index.html')

    with open(html_file, 'w') as f:
        # HTML header
        f.write('<!DOCTYPE html>\n')
        f.write('<html>\n<head>\n')
        f.write(f'<title>{data["project_name"]}</title>\n')
        f.write('<style>\n')
        f.write('body { font-family: Arial, sans-serif; margin: 20px; background: #f0f0f0; }\n')
        f.write('h1 { color: #333; }\n')
        f.write('.gallery { display: grid; grid-template-columns: repeat(auto-fill, minmax(220px, 1fr)); gap: 20px; }\n')
        f.write('.photo { background: white; padding: 10px; border-radius: 8px; box-shadow: 0 2px 4px rgba(0,0,0,0.1); }\n')
        f.write('.photo img { width: 100%; height: 200px; object-fit: cover; border-radius: 4px; }\n')
        f.write('.photo h3 { margin: 10px 0 5px 0; font-size: 16px; }\n')
        f.write('.photo p { margin: 5px 0; font-size: 13px; color: #666; }\n')
        f.write('.tags { margin-top: 10px; }\n')
        f.write('.tag { display: inline-block; background: #007bff; color: white; padding: 3px 8px; border-radius: 3px; font-size: 11px; margin-right: 5px; }\n')
        f.write('</style>\n')
        f.write('</head>\n<body>\n')

        # Title and header
        f.write(f'<h1>{data["project_name"]}</h1>\n')
        f.write(f'<p>By {data["author"]} | {len(data["images"])} photos</p>\n')
        f.write('<hr>\n')

        # Gallery grid
        f.write('<div class="gallery">\n')

        for img_data in data['images']:
            filename = img_data['filename']
            thumb_path = f'thumbnails/{filename}'

            f.write('<div class="photo">\n')
            f.write(f'<img src="{thumb_path}" alt="{img_data["title"]}">\n')
            f.write(f'<h3>{img_data["title"]}</h3>\n')
            f.write(f'<p><strong>Location:</strong> {img_data["location"]}</p>\n')
            f.write(f'<p><strong>Date:</strong> {img_data["date"]}</p>\n')

            # Tags
            f.write('<div class="tags">\n')
            for tag in img_data['tags']:
                f.write(f'<span class="tag">{tag}</span>\n')
            f.write('</div>\n')

            f.write('</div>\n')

        f.write('</div>\n')

        # Footer
        f.write('<hr>\n')
        f.write(f'<p style="text-align: center; color: #999;">Generated by Gallery Generator</p>\n')
        f.write('</body>\n</html>\n')

    print(f"\nGallery created: {html_file}")
    print(f"Open {html_file} in a web browser to view")

# Generate gallery
generate_gallery('data/photo_project.json')
\end{lstlisting}

\task{Data-Driven Batch Processing}

Combine CSV/JSON processing with image manipulation.

\textbf{Create \texttt{data\_driven\_processor.py}:}
\begin{lstlisting}
"""
Advanced data-driven image processor
"""
import json
import csv
from PIL import Image, ImageDraw, ImageFont
import os

class ImageProcessor:
    """Data-driven image processing engine."""

    def __init__(self):
        self.processed_count = 0
        self.error_count = 0
        self.log = []

    def grayscale(self, img, params=None):
        """Convert to grayscale."""
        width, height = img.size
        pixels = img.load()
        result = Image.new("RGB", (width, height))
        result_pixels = result.load()

        for y in range(height):
            for x in range(width):
                r, g, b = pixels[x, y]
                gray = int(0.299 * r + 0.587 * g + 0.114 * b)
                result_pixels[x, y] = (gray, gray, gray)

        return result

    def resize(self, img, params):
        """Resize image."""
        width, height = params['width'], params['height']
        return img.resize((width, height))

    def add_text(self, img, params):
        """Add text overlay."""
        img_copy = img.copy()
        draw = ImageDraw.Draw(img_copy)

        text = params.get('text', 'Sample Text')
        position = params.get('position', 'bottom')
        color = tuple(params.get('color', [255, 255, 255]))

        # Draw background bar
        if position == 'bottom':
            y = img_copy.height - 40
        else:
            y = 0

        draw.rectangle([0, y, img_copy.width, y + 40], fill=(0, 0, 0))
        draw.text((10, y + 10), text, fill=color)

        return img_copy

    def process_json_file(self, json_file):
        """Process images based on JSON configuration."""
        with open(json_file, 'r') as f:
            config = json.load(f)

        for item in config.get('tasks', []):
            try:
                input_file = item['input']
                output_file = item['output']
                operations = item['operations']

                img = Image.open(input_file)

                for op in operations:
                    op_name = op['name']
                    op_params = op.get('params', {})

                    if hasattr(self, op_name):
                        method = getattr(self, op_name)
                        img = method(img, op_params)

                img.save(output_file)
                self.log.append(f"[OK] Processed: {input_file} -> {output_file}")
                self.processed_count += 1

            except Exception as e:
                self.log.append(f"[X] Error: {input_file} - {e}")
                self.error_count += 1

        return self.get_report()

    def get_report(self):
        """Generate processing report."""
        report = []
        report.append("="*60)
        report.append("PROCESSING REPORT")
        report.append("="*60)
        report.append(f"Processed: {self.processed_count}")
        report.append(f"Errors: {self.error_count}")
        report.append("\nDetails:")
        report.extend(self.log)
        report.append("="*60)
        return "\n".join(report)

# Create sample JSON configuration
config = {
    "project": "Batch Processing Demo",
    "tasks": [
        {
            "input": "images/photo1.jpg",
            "output": "output/photo1_processed.jpg",
            "operations": [
                {"name": "resize", "params": {"width": 800, "height": 600}},
                {"name": "grayscale"},
                {"name": "add_text", "params": {"text": "Photo 1", "position": "bottom"}}
            ]
        },
        {
            "input": "images/photo2.jpg",
            "output": "output/photo2_processed.jpg",
            "operations": [
                {"name": "resize", "params": {"width": 640, "height": 480}},
                {"name": "add_text", "params": {"text": "Photo 2", "position": "bottom"}}
            ]
        }
    ]
}

# Save configuration
with open('data/processing_config.json', 'w') as f:
    json.dump(config, f, indent=2)

print("Created data/processing_config.json")

# Process images
processor = ImageProcessor()
report = processor.process_json_file('data/processing_config.json')

print("\n" + report)

# Save report
with open('output/processing_report.txt', 'w') as f:
    f.write(report)

print("\nReport saved to output/processing_report.txt")
\end{lstlisting}

\task{Image Metadata Extractor}

Extract and save image metadata to structured files.

\textbf{Create \texttt{metadata\_extractor.py}:}
\begin{lstlisting}
"""
Extract image metadata and save to CSV/JSON
"""
from PIL import Image
import os
import json
import csv
from datetime import datetime

def extract_metadata(image_path):
    """Extract metadata from an image."""
    img = Image.open(image_path)

    metadata = {
        'filename': os.path.basename(image_path),
        'path': image_path,
        'width': img.width,
        'height': img.height,
        'format': img.format,
        'mode': img.mode,
        'size_bytes': os.path.getsize(image_path),
        'megapixels': round((img.width * img.height) / 1_000_000, 2)
    }

    # Calculate average color
    pixels = img.load()
    total_r, total_g, total_b = 0, 0, 0
    pixel_count = img.width * img.height

    for y in range(img.height):
        for x in range(img.width):
            r, g, b = pixels[x, y]
            total_r += r
            total_g += g
            total_b += b

    metadata['avg_red'] = round(total_r / pixel_count, 2)
    metadata['avg_green'] = round(total_g / pixel_count, 2)
    metadata['avg_blue'] = round(total_b / pixel_count, 2)
    metadata['brightness'] = round((metadata['avg_red'] +
                                   metadata['avg_green'] +
                                   metadata['avg_blue']) / 3, 2)

    return metadata

def scan_directory(directory):
    """Scan directory for images and extract metadata."""
    image_extensions = ('.jpg', '.jpeg', '.png', '.bmp', '.gif')
    image_files = [f for f in os.listdir(directory)
                   if f.lower().endswith(image_extensions)]

    metadata_list = []

    print(f"Scanning {directory}...")
    for filename in image_files:
        path = os.path.join(directory, filename)
        try:
            metadata = extract_metadata(path)
            metadata_list.append(metadata)
            print(f"  [OK] {filename}")
        except Exception as e:
            print(f"  [X] {filename}: {e}")

    return metadata_list

def save_to_csv(metadata_list, output_file):
    """Save metadata to CSV file."""
    if not metadata_list:
        return

    fieldnames = metadata_list[0].keys()

    with open(output_file, 'w', newline='') as f:
        writer = csv.DictWriter(f, fieldnames=fieldnames)
        writer.writeheader()
        writer.writerows(metadata_list)

    print(f"\nCSV saved to: {output_file}")

def save_to_json(metadata_list, output_file):
    """Save metadata to JSON file."""
    output = {
        'scan_date': datetime.now().isoformat(),
        'image_count': len(metadata_list),
        'images': metadata_list
    }

    with open(output_file, 'w') as f:
        json.dump(output, f, indent=2)

    print(f"JSON saved to: {output_file}")

# Scan images directory
metadata = scan_directory('images')

# Save to both formats
save_to_csv(metadata, 'data/image_metadata.csv')
save_to_json(metadata, 'data/image_metadata.json')

# Display statistics
if metadata:
    print("\n" + "="*60)
    print("STATISTICS")
    print("="*60)
    print(f"Total images: {len(metadata)}")
    print(f"Average width: {sum(m['width'] for m in metadata) / len(metadata):.1f}")
    print(f"Average height: {sum(m['height'] for m in metadata) / len(metadata):.1f}")
    print(f"Average megapixels: {sum(m['megapixels'] for m in metadata) / len(metadata):.2f}")
    print(f"Average brightness: {sum(m['brightness'] for m in metadata) / len(metadata):.1f}/255")

    # Largest image
    largest = max(metadata, key=lambda m: m['width'] * m['height'])
    print(f"\nLargest image: {largest['filename']}")
    print(f"  {largest['width']}x{largest['height']} ({largest['megapixels']} MP)")
\end{lstlisting}

\task{Mini-Project: Automated Photo Album}

Create a complete photo album system with metadata and gallery.

\textbf{Create \texttt{photo\_album.py}:}
\begin{lstlisting}
"""
Complete photo album system
"""
import json
from PIL import Image, ImageDraw
import os
from datetime import datetime

class PhotoAlbum:
    """Complete photo album management system."""

    def __init__(self, name):
        self.name = name
        self.photos = []
        self.metadata_file = f'data/{name}_album.json'

    def add_photo(self, filename, title, description, tags=None):
        """Add a photo to the album."""
        photo_data = {
            'id': len(self.photos) + 1,
            'filename': filename,
            'title': title,
            'description': description,
            'tags': tags or [],
            'added_date': datetime.now().isoformat()
        }
        self.photos.append(photo_data)

    def save_metadata(self):
        """Save album metadata to JSON."""
        data = {
            'album_name': self.name,
            'created': datetime.now().isoformat(),
            'photo_count': len(self.photos),
            'photos': self.photos
        }

        with open(self.metadata_file, 'w') as f:
            json.dump(data, f, indent=2)

        print(f"Metadata saved to {self.metadata_file}")

    def load_metadata(self):
        """Load album metadata from JSON."""
        try:
            with open(self.metadata_file, 'r') as f:
                data = json.load(f)
                self.photos = data['photos']
            print(f"Loaded {len(self.photos)} photos")
        except FileNotFoundError:
            print("No metadata file found")

    def generate_thumbnails(self, thumb_dir='output/thumbnails'):
        """Generate thumbnails for all photos."""
        if not os.path.exists(thumb_dir):
            os.makedirs(thumb_dir)

        for photo in self.photos:
            try:
                img = Image.open(f"images/{photo['filename']}")
                img.thumbnail((200, 200))
                thumb_path = os.path.join(thumb_dir, photo['filename'])
                img.save(thumb_path)
                print(f"  [OK] {photo['filename']}")
            except Exception as e:
                print(f"  [X] {photo['filename']}: {e}")

    def create_contact_sheet(self, output_file='output/contact_sheet.jpg'):
        """Create a contact sheet with all photos."""
        thumb_size = 200
        padding = 10
        cols = 4
        rows = (len(self.photos) + cols - 1) // cols

        width = cols * (thumb_size + padding) + padding
        height = rows * (thumb_size + padding) + padding + 50

        contact = Image.new('RGB', (width, height), (255, 255, 255))
        draw = ImageDraw.Draw(contact)

        # Title
        draw.text((padding, padding), f'{self.name} - {len(self.photos)} photos',
                  fill=(0, 0, 0))

        y_offset = 50
        for idx, photo in enumerate(self.photos):
            try:
                img = Image.open(f"images/{photo['filename']}")
                img.thumbnail((thumb_size, thumb_size))

                row = idx // cols
                col = idx % cols

                x = col * (thumb_size + padding) + padding
                y = row * (thumb_size + padding) + padding + y_offset

                contact.paste(img, (x, y))

            except Exception as e:
                print(f"Error adding {photo['filename']}: {e}")

        contact.save(output_file)
        print(f"Contact sheet saved to {output_file}")

    def search_by_tag(self, tag):
        """Search photos by tag."""
        results = [p for p in self.photos if tag in p['tags']]
        return results

    def generate_report(self):
        """Generate text report."""
        report = []
        report.append("="*60)
        report.append(f"PHOTO ALBUM: {self.name}")
        report.append("="*60)
        report.append(f"Total photos: {len(self.photos)}")
        report.append("")

        # List all tags
        all_tags = []
        for photo in self.photos:
            all_tags.extend(photo['tags'])

        unique_tags = set(all_tags)
        report.append(f"Tags used ({len(unique_tags)}):")
        for tag in sorted(unique_tags):
            count = all_tags.count(tag)
            report.append(f"  {tag}: {count} photos")

        report.append("")
        report.append("Photos:")
        for photo in self.photos:
            report.append(f"\n  {photo['id']}. {photo['title']}")
            report.append(f"     File: {photo['filename']}")
            report.append(f"     Description: {photo['description']}")
            report.append(f"     Tags: {', '.join(photo['tags'])}")

        report.append("\n" + "="*60)
        return "\n".join(report)

# Create sample album
album = PhotoAlbum("My Vacation 2024")

# Add photos
album.add_photo("photo1.jpg", "Beach Sunset",
                "Beautiful sunset at the beach",
                ["beach", "sunset", "vacation"])
album.add_photo("photo2.jpg", "Mountain Hike",
                "Hiking in the mountains",
                ["mountain", "hiking", "nature"])
album.add_photo("photo3.jpg", "City Night",
                "City lights at night",
                ["city", "night", "urban"])

# Save metadata
album.save_metadata()

# Generate outputs
print("\nGenerating thumbnails...")
album.generate_thumbnails()

print("\nCreating contact sheet...")
album.create_contact_sheet()

# Generate report
report = album.generate_report()
print("\n" + report)

# Save report
with open('output/album_report.txt', 'w') as f:
    f.write(report)

print("\nAlbum system complete!")
\end{lstlisting}

\section{Bonus Challenges}

\subsection*{Challenge 1: XML Processing}
Add XML support for image metadata alongside JSON and CSV.

\subsection*{Challenge 2: Database Integration}
Use SQLite to store and query image metadata.

\subsection*{Challenge 3: API Integration}
Fetch image data from a web API and process it.

\subsection*{Challenge 4: Advanced Gallery}
Create a gallery with sorting, filtering, and search features.

\subsection*{Challenge 5: Automated Backup}
Create a system that backs up images and their metadata to a structured archive.

\section{Key Concepts}

\textbf{Data processing concepts:}
\begin{itemize}
    \item \textbf{CSV:} Simple tabular data format
    \item \textbf{JSON:} Hierarchical data structure
    \item \textbf{Metadata:} Data about data
    \item \textbf{Serialization:} Converting data to storable format
    \item \textbf{Data-driven:} Using data to control program behavior
    \item \textbf{Batch processing:} Automated repetitive tasks
\end{itemize}

\section{Common Issues}

\textbf{JSON parsing errors:}
\begin{itemize}
    \item Check for missing commas or brackets
    \item Ensure proper quoting of strings
    \item Use online JSON validators
\end{itemize}

\textbf{CSV encoding issues:}
\begin{itemize}
    \item Use UTF-8 encoding
    \item Handle commas within data fields
    \item Use csv module for complex files
\end{itemize}

\textbf{Path problems:}
\begin{itemize}
    \item Use \texttt{os.path.join()} for cross-platform paths
    \item Check file existence before processing
    \item Create output directories if needed
\end{itemize}

\section{Checkoff}

Before you leave, show your instructor or TA:
\begin{enumerate}
    \item CSV-driven batch processor working on multiple images
    \item JSON data being read and processed
    \item Generated HTML image gallery
    \item Metadata extraction saving to both CSV and JSON
    \item Explain the difference between CSV and JSON
\end{enumerate}

\section{What's Next}

In Lab 11, we'll create animations by generating sequences of images and combining them into animated GIFs!

\end{document}
