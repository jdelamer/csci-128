\documentclass{cslab}

\usepackage{listings}
\usepackage{xcolor}

% Python code styling
\lstset{
    language=Python,
    basicstyle=\ttfamily\small,
    keywordstyle=\color{blue}\bfseries,
    stringstyle=\color{red},
    commentstyle=\color{green!50!black}\itshape,
    showstringspaces=false,
    breaklines=true,
    frame=single,
    numbers=left,
    numberstyle=\tiny\color{gray},
    tabsize=4
}

% Lab information
\labnumber{05}
\labtitle{Combining Images and Collages}
\course{CSCI 128: Introduction to Computer Science}
\courseshort{CSCI 128}
\semester{Fall 2024}
\timelimit{2 hours}
\logoimage{LogoWordsBottom.png}

\begin{document}

\maketitle

\begin{objectives}
    \item Combine multiple images into collages
    \item Paste images onto larger canvases
    \item Implement image blending and transparency
    \item Create chromakey (green screen) effects
    \item Work with alpha channels and masks
    \item Design creative photo compositions
    \item Understand image layering techniques
\end{objectives}

\begin{grading}
\textbf{Participation-Based Grading:} Complete the tasks during the lab session and get checked off by the instructor or TA.

\textbf{To receive credit:}
\begin{itemize}
    \item Attend the entire lab session
    \item Complete Tasks 1-6 (Bonus is optional)
    \item Demonstrate your collage programs
    \item Get checked off before leaving
\end{itemize}
\end{grading}

\section{Introduction}

Today you'll learn to combine multiple images to create collages, compositions, and special effects! You'll work with multiple images simultaneously, understanding how professional photo editing software layers and blends images. By the end, you'll be able to create your own photo montages and even implement green screen effects!

\section{Setup}

\begin{setup}
\textbf{Before you begin:}
\begin{enumerate}
    \item Create a folder: \texttt{csci128/lab05}
    \item Ensure Pillow is installed: \texttt{pip install Pillow}
    \item Gather 5-10 diverse images:
    \begin{itemize}
        \item Different sizes (small logos, medium photos, large backgrounds)
        \item A plain colored background or gradient
        \item Several photos you want to combine
        \item Optional: A green screen image (or take one with phone)
    \end{itemize}
    \item Put images in an \texttt{images} subfolder
\end{enumerate}
\end{setup}

\section{Lab Tasks}

\task{Pasting Images onto a Canvas}

Learn the basic operation of placing one image onto another.

\textbf{Create \texttt{paste\_simple.py}:}
\begin{lstlisting}
from PIL import Image

# Create a canvas (background)
canvas_width = 800
canvas_height = 600
canvas = Image.new("RGB", (canvas_width, canvas_height), (255, 255, 255))

# Load images to paste
img1 = Image.open("images/photo1.jpg")
img2 = Image.open("images/photo2.jpg")

# Resize images to fit nicely
img1 = img1.resize((200, 200))
img2 = img2.resize((200, 200))

# Paste images at specific positions
canvas.paste(img1, (50, 50))     # Top-left
canvas.paste(img2, (550, 50))    # Top-right

# Save result
canvas.save("simple_collage.jpg")
print("Created simple_collage.jpg")
canvas.show()
\end{lstlisting}

\textbf{Experiments:}
\begin{enumerate}
    \item Add more images to different positions
    \item Try different canvas colors: black (0,0,0), blue (100,150,200)
    \item What happens if you paste outside the canvas bounds?
    \item Overlap images by pasting to nearby positions
\end{enumerate}

\begin{note}
The \texttt{paste()} method takes an image and a position tuple (x, y) representing the top-left corner where the image will be placed.
\end{note}

\task{Creating a Grid Collage}

Arrange multiple images in a neat grid pattern.

\textbf{Create \texttt{grid\_collage.py}:}
\begin{lstlisting}
from PIL import Image
import os

# Configuration
images_per_row = 3
image_size = 200
padding = 10

# Load images
image_folder = "images"
image_files = [f for f in os.listdir(image_folder)
               if f.endswith(('.jpg', '.png', '.jpeg'))][:9]

# Calculate canvas size
canvas_width = (image_size + padding) * images_per_row + padding
rows = (len(image_files) + images_per_row - 1) // images_per_row
canvas_height = (image_size + padding) * rows + padding

# Create canvas
canvas = Image.new("RGB", (canvas_width, canvas_height), (50, 50, 50))

# Place images in grid
for idx, img_file in enumerate(image_files):
    img = Image.open(os.path.join(image_folder, img_file))

    # Resize to square
    img = img.resize((image_size, image_size))

    # Calculate grid position
    col = idx % images_per_row
    row = idx // images_per_row

    x = padding + col * (image_size + padding)
    y = padding + row * (image_size + padding)

    # Paste image
    canvas.paste(img, (x, y))
    print(f"Placed {img_file} at ({x}, {y})")

canvas.save("grid_collage.jpg")
print(f"\nCreated grid_collage.jpg with {len(image_files)} images")
canvas.show()
\end{lstlisting}

\textbf{Customizations to try:}
\begin{enumerate}
    \item Change images per row (2, 4, 5)
    \item Adjust padding between images
    \item Try different background colors
    \item Add white borders around each image
    \item Make rectangular images instead of squares
\end{enumerate}

\task{Image Blending}

Blend two images together for artistic effects.

\textbf{Create \texttt{blend\_images.py}:}
\begin{lstlisting}
from PIL import Image

# Load two images
img1 = Image.open("images/photo1.jpg")
img2 = Image.open("images/photo2.jpg")

# They must be the same size for blending
# Resize both to common dimensions
size = (600, 400)
img1 = img1.resize(size)
img2 = img2.resize(size)

# Get pixel data
width, height = size
pixels1 = img1.load()
pixels2 = img2.load()

# Create blended image
blended = Image.new("RGB", size)
blended_pixels = blended.load()

# Blend with 50-50 mix
alpha = 0.5  # Blend ratio (0.0 = all img1, 1.0 = all img2)

for y in range(height):
    for x in range(width):
        r1, g1, b1 = pixels1[x, y]
        r2, g2, b2 = pixels2[x, y]

        # Weighted average
        new_r = int(r1 * (1 - alpha) + r2 * alpha)
        new_g = int(g1 * (1 - alpha) + g2 * alpha)
        new_b = int(b1 * (1 - alpha) + b2 * alpha)

        blended_pixels[x, y] = (new_r, new_g, new_b)

blended.save("blended_50_50.jpg")
print("Created blended_50_50.jpg")
blended.show()

# Try different blend ratios
for alpha in [0.25, 0.75]:
    blended2 = Image.new("RGB", size)
    blended2_pixels = blended2.load()

    for y in range(height):
        for x in range(width):
            r1, g1, b1 = pixels1[x, y]
            r2, g2, b2 = pixels2[x, y]

            new_r = int(r1 * (1 - alpha) + r2 * alpha)
            new_g = int(g1 * (1 - alpha) + g2 * alpha)
            new_b = int(b1 * (1 - alpha) + b2 * alpha)

            blended2_pixels[x, y] = (new_r, new_g, new_b)

    blended2.save(f"blended_{int(alpha*100)}.jpg")
    print(f"Created blended_{int(alpha*100)}.jpg")
\end{lstlisting}

\textbf{Experiments:}
\begin{enumerate}
    \item Try alpha values: 0.1, 0.3, 0.5, 0.7, 0.9
    \item Blend a portrait with a landscape
    \item Create a "dissolve" effect with multiple alpha values
    \item Blend three images (blend result of first two with third)
\end{enumerate}

\begin{hint}
Pillow has a built-in blend function: \texttt{Image.blend(img1, img2, alpha)}. Compare your manual version with the built-in!
\end{hint}

\task{Gradient Blending}

Create smooth transitions between images using position-based blending.

\textbf{Create \texttt{gradient\_blend.py}:}
\begin{lstlisting}
from PIL import Image

# Load images
img1 = Image.open("images/photo1.jpg")
img2 = Image.open("images/photo2.jpg")

# Resize to same dimensions
size = (800, 400)
img1 = img1.resize(size)
img2 = img2.resize(size)

width, height = size
pixels1 = img1.load()
pixels2 = img2.load()

# Create gradient blended image
gradient_blend = Image.new("RGB", size)
gb_pixels = gradient_blend.load()

# Horizontal gradient blend (left to right)
for y in range(height):
    for x in range(width):
        # Alpha changes from 0 (left) to 1 (right)
        alpha = x / width

        r1, g1, b1 = pixels1[x, y]
        r2, g2, b2 = pixels2[x, y]

        new_r = int(r1 * (1 - alpha) + r2 * alpha)
        new_g = int(g1 * (1 - alpha) + g2 * alpha)
        new_b = int(b1 * (1 - alpha) + b2 * alpha)

        gb_pixels[x, y] = (new_r, new_g, new_b)

gradient_blend.save("gradient_blend_horizontal.jpg")
print("Created gradient_blend_horizontal.jpg")
gradient_blend.show()

# Vertical gradient blend
vertical_blend = Image.new("RGB", size)
vb_pixels = vertical_blend.load()

for y in range(height):
    for x in range(width):
        # Alpha changes from 0 (top) to 1 (bottom)
        alpha = y / height

        r1, g1, b1 = pixels1[x, y]
        r2, g2, b2 = pixels2[x, y]

        new_r = int(r1 * (1 - alpha) + r2 * alpha)
        new_g = int(g1 * (1 - alpha) + g2 * alpha)
        new_b = int(b1 * (1 - alpha) + b2 * alpha)

        vb_pixels[x, y] = (new_r, new_g, new_b)

vertical_blend.save("gradient_blend_vertical.jpg")
print("Created gradient_blend_vertical.jpg")
vertical_blend.show()
\end{lstlisting}

\textbf{Try these variations:}
\begin{enumerate}
    \item Diagonal blend (use both x and y)
    \item Radial blend (circular from center)
    \item Custom blend patterns (checkerboard, stripes)
\end{enumerate}

\task{Chromakey (Green Screen) Effect}

Replace a specific color (like green) with another image!

\textbf{Create \texttt{chromakey.py}:}
\begin{lstlisting}
from PIL import Image

# Load foreground (person on green screen) and background
foreground = Image.open("images/greenscreen.jpg")
background = Image.open("images/background.jpg")

# Resize to match
size = foreground.size
background = background.resize(size)

width, height = size
fg_pixels = foreground.load()
bg_pixels = background.load()

# Create result image
result = Image.new("RGB", size)
result_pixels = result.load()

# Define "green" threshold
green_threshold = 100

for y in range(height):
    for x in range(width):
        r, g, b = fg_pixels[x, y]

        # Check if pixel is "greenish"
        # Green is dominant and significantly higher than red and blue
        if g > green_threshold and g > r and g > b:
            # Replace with background
            result_pixels[x, y] = bg_pixels[x, y]
        else:
            # Keep foreground
            result_pixels[x, y] = (r, g, b)

result.save("chromakey_result.jpg")
print("Created chromakey_result.jpg")
result.show()
\end{lstlisting}

\textbf{Improve the chromakey:}
\begin{enumerate}
    \item Adjust green\_threshold for better detection
    \item Check if green is much larger than r and b: \texttt{g > r + 50 and g > b + 50}
    \item Try with blue screen (blue > red and blue > green)
    \item Soften edges by partially blending border pixels
\end{enumerate}

\begin{note}
Professional chromakey is more sophisticated, using color distance in HSV space and edge softening. But this simple version demonstrates the core concept!
\end{note}

\task{Picture Frame Effect}

Add decorative frames and borders to images.

\textbf{Create \texttt{picture\_frame.py}:}
\begin{lstlisting}
from PIL import Image

def add_frame(img, frame_width, frame_color):
    """Add a colored frame around an image"""
    old_width, old_height = img.size

    # New dimensions include frame
    new_width = old_width + 2 * frame_width
    new_height = old_height + 2 * frame_width

    # Create framed image with frame color
    framed = Image.new("RGB", (new_width, new_height), frame_color)

    # Paste original image in center
    framed.paste(img, (frame_width, frame_width))

    return framed

def add_fancy_frame(img, outer_width, inner_width,
                    outer_color, inner_color):
    """Add a two-color decorative frame"""
    # Add inner frame
    framed = add_frame(img, inner_width, inner_color)

    # Add outer frame
    framed = add_frame(framed, outer_width, outer_color)

    return framed

# Load image
img = Image.open("images/photo1.jpg")

# Simple frame
simple_frame = add_frame(img, 20, (255, 215, 0))  # Gold
simple_frame.save("simple_frame.jpg")
print("Created simple_frame.jpg")

# Fancy frame
fancy_frame = add_fancy_frame(img, 15, 5,
                              (139, 69, 19),   # Brown
                              (255, 215, 0))   # Gold
fancy_frame.save("fancy_frame.jpg")
print("Created fancy_frame.jpg")
fancy_frame.show()

# Multiple images with frames in a grid
canvas = Image.new("RGB", (1200, 400), (50, 50, 50))
images = ["images/photo1.jpg", "images/photo2.jpg",
          "images/photo3.jpg"]

x_pos = 50
for img_file in images[:3]:
    img = Image.open(img_file)
    img = img.resize((300, 300))
    framed = add_frame(img, 10, (255, 255, 255))
    canvas.paste(framed, (x_pos, 50))
    x_pos += 350

canvas.save("framed_gallery.jpg")
print("Created framed_gallery.jpg")
canvas.show()
\end{lstlisting}

\task{Mini-Project: Create a Collage}

Design your own creative photo collage!

\textbf{Create \texttt{my\_collage.py}:}

Your collage should include:
\begin{enumerate}
    \item A large canvas (at least 1200x800)
    \item At least 5 different images
    \item Images at different sizes
    \item Some overlapping elements
    \item A creative layout (not just a grid)
    \item Optional: frames, borders, or blend effects
\end{enumerate}

\textbf{Ideas for layouts:}
\begin{itemize}
    \item \textbf{Magazine cover:} Large central image with smaller images around
    \item \textbf{Photo strip:} Vertical sequence of images
    \item \textbf{Scattered polaroids:} Random positions and slight rotations
    \item \textbf{Overlap art:} Blend overlapping images
    \item \textbf{Story board:} Sequential images telling a story
\end{itemize}

\textbf{Example structure:}
\begin{lstlisting}
from PIL import Image

# Create large canvas
canvas = Image.new("RGB", (1200, 800), (240, 240, 240))

# Load and prepare images
img1 = Image.open("images/photo1.jpg").resize((500, 400))
img2 = Image.open("images/photo2.jpg").resize((300, 300))
img3 = Image.open("images/photo3.jpg").resize((250, 250))
# ... more images ...

# Paste in creative positions
canvas.paste(img1, (50, 200))
canvas.paste(img2, (600, 50))
canvas.paste(img3, (700, 450))
# ... more placements ...

# Add frames or effects
# ... your creative touches ...

canvas.save("my_collage.jpg")
canvas.show()
\end{lstlisting}

\section{Bonus Challenges}

\subsection*{Challenge 1: Vignette Effect}
Add a vignette (darkened edges) to an image by blending it with a black image using radial gradient.

\subsection*{Challenge 2: Before/After Slider}
Create a split image showing before/after of a filtered image with a vertical divider line.

\subsection*{Challenge 3: Photo Mosaic}
Create a large image made up of many tiny thumbnail images arranged in a grid.

\subsection*{Challenge 4: Watermark}
Add a semi-transparent text or logo watermark to an image.

\subsection*{Challenge 5: Animated Blend}
Create a series of images with gradually changing blend ratios (0.0, 0.1, 0.2, ... 1.0) for animation.

\section{Key Concepts}

\textbf{Important techniques covered:}
\begin{itemize}
    \item \textbf{Pasting:} Placing one image onto another at specific coordinates
    \item \textbf{Blending:} Weighted average of two images' pixels
    \item \textbf{Alpha:} Blend ratio (0.0 to 1.0)
    \item \textbf{Canvas:} Background image to paste onto
    \item \textbf{Chromakey:} Color-based image replacement
    \item \textbf{Compositing:} Combining multiple images with various techniques
\end{itemize}

\section{Common Issues}

\textbf{Images different sizes:}
\begin{itemize}
    \item Use \texttt{resize()} to make them compatible
    \item Crop to same dimensions first
    \item Paste smaller onto larger without blending
\end{itemize}

\textbf{Pasting outside canvas bounds:}
\begin{itemize}
    \item Check position + image size < canvas size
    \item Pillow will crop automatically but might not look as expected
\end{itemize}

\textbf{Chromakey not working:}
\begin{itemize}
    \item Adjust threshold values
    \item Ensure green screen is evenly lit
    \item Check color values with pixel inspection
    \item Try different color comparison formulas
\end{itemize}

\section{Checkoff}

Before you leave, show your instructor or TA:
\begin{enumerate}
    \item Your grid collage with at least 6 images
    \item Your blended images showing different alpha values
    \item Your chromakey effect (or gradient blend if no green screen)
    \item Your creative collage project
    \item Explain how blending works (weighted average)
\end{enumerate}

\section{What's Next?}

In Lab 06, we'll switch from images to sounds! You'll learn how digital audio works and create your own sound effects. Bring headphones!

\end{document}
